\documentclass[10pt]{article}
\usepackage[utf8]{inputenc}
\usepackage[T1]{fontenc}
\usepackage{amsmath}
\usepackage{amsfonts}
\usepackage{amssymb}
\usepackage[version=4]{mhchem}
\usepackage{stmaryrd}
\usepackage{tikz}
%\usepackage{enumitem}
\usepackage[inline]{enumitem}

%\input{.tex/preamble}
%\input{.tex/macros}
%\input{.tex/letterfonts}

\title{UW - Math 632 \\
Stachastic Processes \\
Quiz 1}

\author{Guy Matz}
\date{\today}


\begin{document}

    \begin{figure}
      \begin{tikzpicture}
      \node[state] (s1) {1};
      \node[state, below of=s1] (s2) {2};
      \node[state, right of=s2] (s3) {3};
      \draw
      (s1) edge[bend left] node{} (s3)
      (s3) edge[bend left] node{} (s1)
      (s2) edge[bend right] node{} (s3)
      (s2) edge[loop left] node{} (s2)
      ;
      \end{tikzpicture}
      \caption{problem 1}
    \end{figure}

\begin{enumerate}
  \item Below is a transition diagram for a time-homogeneous Markov
    chain with state space $\{1,2,3\}$
. The diagram depicts a transition from state i to state j with a directed arrow from i to j if and only if the probability of this transition between successive time steps is non-zero. Which of the following matrices could be a probability transition matrix that corresponds to the diagram below?
  0,0,1 \\
  0,1/3,2/3 \\
  1,0,0

  \item Below is a transition diagram for a time-homogeneous Markov
    chain $\{X_n : n \geq 0 \}$ with state space $\{1,2\}$ and 
    $P(X_0=1)-1$.  . The diagram depicts a transition from state i to
    state j with a directed arrow from i to j if and only if the
    probability of this transition between successive time steps is
    non-zero. The probability of such transitions is denoted next to its
    corresponding arrow. If $q = 0.6$, what is the probability that 
    $X_2 = 2$?
    \begin{figure}
      \begin{tikzpicture}
      \node[state] (s1) {1};
      \node[state, right of=s1] (s2) {2};
      \draw
      (s1) edge[loop above] node{q} (s1)
      (s1) edge[bend left, above] node{1-q} (s2)
      (s2) edge[loop above] node{1} (s2)
      ;
      \end{tikzpicture}
      \caption{problem 2 and 3}
    \end{figure}
    $P(X_2=2) = 0.64$

  \item Abbove is a transition diagram for a time-homogeneous Markov
    chain $\{X_n : n \geq 0 \}$ with state space $\{1,2\}$ and 
    $P(X_0=1)-1$.  . The diagram depicts a transition from state i to
    state j with a directed arrow from i to j if and only if the
    probability of this transition between successive time steps is
    non-zero. The probability of such transitions is denoted next to its
    corresponding arrow. If $q = 0.7$, what is the probability that 
    there is a positive integer n such that$X_n = 2$?

    P=1

  \item Consider a time-homogeneous Markov chain $X_n(n=0,1,2, \ldots)$ with state space $\{1,2,3\}$ and probability transition matrix
$$
\mathrm{P}=\left[\begin{array}{lll}
p(1,1) & p(1,2) & p(1,3) \\
p(2,1) & p(2,2) & p(2,3) \\
p(3,1) & p(3,2) & p(3,3)
\end{array}\right]
$$
so that $p(i, j)$ denotes the probability $P\left(X_{n+1}=j \mid
X_n=i\right)$. Which of the following quantities is equal to the
multi-step probability $P\left(X_{n+4}=3 \mid X_n=1\right)$ ?
Assume $\mathrm{A}_{\mathrm{ij}}$ denotes the entry in the ith row and jth
column of any matrix $\mathrm{A}$.
$\left(\mathrm{P}^4\right)_{13} \checkmark$ \\
$p(3,1)^4$ \\
$p(1,3)^4$ \\
$\left(\mathrm{P}^4\right)_{31}$

Answer A: $\left(\mathrm{P}^4\right)_{13}$

  \item Consider the Simple Symmetric Random Walk (SSRW)
    $\left\{S_n: n \geq 0\right\}$ starting at 0 . Let $T$ be the first
    time that $S_n$ visits 2023, i.e. $T=\inf \left\{n \geq 0: S_n=2023\right\}$, then $T$ is a stopping time.

True TRUE!
False

  \item Consider the Simple Symmetric Random Walk (SSRW)
    $\left\{S_n: n \geq 0\right\}$ starting at 0 . Let $T$ be the first
    time that $S_n$ visits 2023, i.e. $T=\inf \left\{n \geq 0: S_n=2023\right\}$, then $T$ is a stopping time.

True 
False FALSE!  That's looking into the future!!

  \item Consider a Markov chain $X_n$ with state space $\mathcal{S}$ and probability transition function $p$. For $x \in \mathcal{S}$, which of the following are equivalent to the statement that $x$ is transient:
$E_x\left[N_x\right]<\infty$ where $N_x$ is the number of times that $X_n$ visits $\mathrm{x} \checkmark$
$P_x\left(\left\{T_x^k<\infty\right.\right.$ for all $\left.\left.k \geq 1\right\}\right)=0$
$\sum_{n=1}^{\infty} p^{(n)}(x, x)<\infty \checkmark$


\item Below is a transition diagram for a time-homogeneous Markov chain 
  $X_n$ with state space  $\{1,2,3\}$
. The diagram depicts a transition from state i to state j with a directed arrow from i to j if and only if the probability of this transition between successive time steps is non-zero. Which of the following is a closed and irreducible set with respect to $X_n$ ?
\begin{enumerate}
  \item $\{1\}$
  \item $\{1,3\} \checkmark$
  \item $\{2\}$
  \item $\{1,2,3\}$
  \item $\{1,2\}$
  \item $\{3\}$
  \item $\{2,3\}$
\end{enumerate}

\item Consider a Markov chain with state space $\mathcal{S}=\{1,2,3\}$, transition matrix $\mathrm{P}=\left[\begin{array}{ccc}1-a-b & a & b \\ c & 1-c-d & d \\ a & b & 1-a-b\end{array}\right]$ and stationary distribution $\pi:=\left[\begin{array}{lll}\pi(1) & \pi(2) & \pi(3)\end{array}\right]$. Which of the following must hold?
$\pi(1)+\pi(2)+\pi(3)=1 \checkmark$ \\
The 2nd entry of $\mathrm{P} \pi^{\mathrm{T}}$ is 1 \\
The 3rd entry of $\pi \mathrm{P}$ is $\pi(3) \checkmark$ \\
The 1st entry of $\mathrm{P} \pi^T$ is $\pi(1)$
\end{enumerate}
\end{document}
