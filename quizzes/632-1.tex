\documentclass[10pt]{article}
\usepackage[utf8]{inputenc}
\usepackage[T1]{fontenc}
\usepackage{amsmath}
\usepackage{amsfonts}
\usepackage{amssymb}
\usepackage[version=4]{mhchem}
\usepackage{stmaryrd}
\usepackage{tikz}
%\usepackage{enumitem}
\usepackage[inline]{enumitem}

%\input{.tex/preamble}
%\input{.tex/macros}
%\input{.tex/letterfonts}

\title{UW - Math 632 \\
Stachastic Processes \\
Quiz 1}

\author{Guy Matz}
\date{\today}


\begin{document}

\begin{enumerate}
  \item Suppose $X$ is a geometric random variable with success
    probability $p=0.4$ and probability mass function:
    \[ P(X=n)=p(1-p)^{n-1} \]
    Compute the expectation of $X$.

    \[ E[X] = \frac{1}{p} = \frac{1}{\frac{4}{10} } = 2.5   \]

\newpage

  \item Which of the following statements are true?
    \begin{enumerate}
      \item Every real-valued random variable has a probability mass function.
      \item Every real-valued random variable has a cumulative distribution function. \checkmark
      \item Every real-valued random variable has a probability distribution. \checkmark
      \item Every real-valued random variable has a probability density function
    \end{enumerate}

\newpage

  \item Let $\{ X_k : k = 0,1,2 \dots \}$ denote a collection of random
    variables such that $X_k$ are independent Binomial random variables
    with parameters $n$ and $p$.
    Which of the following statements are correct?
    \begin{itemize}
		\item $X_k$ is an independent process but not an identically-distributed process
		\item $X_k$ is a renewal process
		\item $X_k$ is a stochastic process \checkmark
		\item $X_k$ is an iid process \checkmark
    \end{itemize}

\newpage

  \item Let $U \sim$ Unif[0,6]. Calculate the expectation of the random
    variable $min(U, 4)$.

    \begin{align*}
      E[min(U, 4)] &= E[min(U, 4) | U \leq 4)P(U \leq 4) +
                              E[min(U,4) | U > 4)P(U > 4) \\
                   &= 2 \cdot  \frac{2}{3} + 4 \cdot  \frac{1}{3}  \\
                   &= \frac{4}{3} + \frac{4}{3}  \\
                   &= 2.667
    \end{align*}

\newpage

  \item Suppose the waiting times $\{X_k, k \geq 1\}$ of customers at
    restaurant A (in minutes) are i.i.d. RVs distributed according to the
    Unif[0,8] distribution. If the waiting time exceeds 4 (minutes), then
    the customer will be served immediately. In the long run, what is the
    rate at which the customers are served? 

    Provide a numerical answer with precision up to 2 decimal places.

    Hint: Question 4 shows how to compute the expectation of the minimum of
    a uniform random variable and a deterministic constant. Then apply the
    Renewal theorem. 

    Let $U \sim$ Unif[0,8].  Then
    \begin{align*}
      E[min(U, 4)] &= E[min(U, 4) | U \leq 4)P(U \leq 4) + 
                              E[min(U,4) | U > 4)P(U > 4) \\
                   &= 2 \cdot  \frac{1}{2} + 4 \cdot  \frac{1}{2}  \\
                   &= 1 + 2  \\
                   &= 3
    \end{align*}

    So 
    \[ \lim_{n \to \infty} \frac{N_t}{n} = \frac{1}{E[X]} = .33 \]

\newpage

  \item Suppose the lifetimes of light bulbs used in a storage room have
    i.i.d. exponential distribution with mean 200 days. Suppose that a
    custodian checks the bulb exactly at noon every day. A burned-out bulb
    is replaced immediately upon discovery. Let $X_k$ denote the length of
    the k-th cycle. $X_k$ is a \underline{geometric} random variable.
    (Fill in the distribution of $X$.) The long-term rate at which bulbs
    are consumed is \underline{1/200.5} ??

    \[ \lim_{t \to \infty} \frac{N}{t} = \frac{1}{\mu} = p = 1-e^
    {-\frac{1}{200}} \]

\newpage

  \item Every iid process $X_n$ satisfies the Markov property.
    \begin{itemize}
      \item True \checkmark
      \item False
    \end{itemize}

\newpage

  \item Consider a Markov chain $X_n$ with state space $S$.  Which of the
    following quantities must be equal to $P(X_4 = x_4 | X_0 = x_0, X_1 = 
    x_1, X_2 = x_2, X_3 = x_3)$ for $x_0, \dots, x_4 \in S$ provided
    $P(X_0=x_0, \dots X_3 = x_3) > 0$?
    \begin{enumerate}
      \item $P(X_4 = x_4 | X_0 = x_0)$
      \item $P(X_4 = x_4 | X_3 = x_3)$ \checkmark
      \item $P(X_4 = x_4 | X_5 = x_5)$
    \end{enumerate}

\newpage

  \item Suppose $X_n$ is a simple random walk with transition
    probabilities $P(X_{n+1} = b|X_n = a)$ equal to 0.5 if $b=a+1$ and
    equal to (1-0.5) if $b=a-1$.  If $P(X_0=0)=1$, what is $(X_3=-1)$

    .375

\end{enumerate}
\end{document}
