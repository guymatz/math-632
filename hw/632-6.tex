\documentclass{article} % This command is used to set the type of document you are working on such as an article, book, or presenation

%\usepackage[margin=1in]{geometry} % This package allows the editing of the page layout. I've set the margins to be 1 inch. 

\usepackage{amsmath, amsfonts}  % The first package allows the use of a large range of mathematical formula, commands, and symbols.  The second gives some useful mathematical fonts.

\usepackage{graphicx}  % This package allows the importing of images
\usepackage{marvosym}  % Lightning!
\usepackage{wasysym}  % Smileys!

\usepackage{bbold}
%This allows us to use the theorem and proof environment 
\usepackage{enumitem}
\usepackage{amsthm}
\theoremstyle{plain}
\newtheorem*{theorem*}{Theorem}
\newtheorem{theorem}{Theorem}

\newtheoremstyle{case}{}{}{}{}{}{:}{ }{}
\theoremstyle{case}
\newtheorem{case}{Case}

%Custom commands.  
\newcommand{\abs}[1]{\left\lvert #1 \right\rvert} %absolute value command

%Custom symbols
\newcommand{\Rb}{\mathbb{R}}

\begin{document}

\begin{center}
\Large{\textbf{Assignment \#6}
            
UW-Madison MATH 632} % Name of course here
\vspace{5pt}
        
\normalsize{  Guy Matz% Your name here
        \\ Due: July 24, 2023}
\vspace{15pt}
\end{center}

\section*{Exercises}%

\begin{enumerate}[label={\fbox{\textbf{Exercise \#\arabic* :}}}]

  \item 
    \begin{enumerate}
    \item Let $X \sim \operatorname{Poisson}(\lambda)$. Find its probability generating function (PGF) $g_X(s)$. Use the PGF to find $E X$.
      \[ \text{ PGF } = g(s)=\sum_{k=0}^{\infty} s^k P(X=k) \]
    The slope of the generating function $g(s)$ at $s=1$ gives the mean EX

    \item  Let $X_1, \ldots, X_n$ be independent with marginal distribution $X_k \sim \operatorname{Poisson}\left(\lambda_k\right)$ for $k=$ $1, \ldots, n$. Let $S=X_1+\cdots+X_n$ denote the sum. Use PGFs to identify the distribution of $S$
  \end{enumerate}
\newpage
  \item Let $0<p<1$. Consider a branching process whose offspring distribution satisfies $\beta_0=1-p$ and $\beta_3=p$. Find the extinction probability.

    \emph{Hint: You solve a third degree polynomial, but one of the roots should be easy to guess. Note that the PGF for the offspring distribution satisfies $g(1)=E\left[1^Z\right]=1$. Hence the problem reduces to finding the roots of a quadratic.}
\newpage

  \item Let $0<p<1$. Consider a population of organisms whose lifecycle goes as follows. A newborn individual has probability $p$ of reaching adulthood. Once an adult, the individual gives birth to exactly two offspring, and then dies. Start with a single adult individual. Find the probability that this population eventually goes extinct.
    \emph{Hint: Consider the process of the adults.}
\newpage

  \item Suppose that phone calls arrive to an office at rate $10 /$ hour Poisson process. Find the probabilities of the following events.

    See Exercise 4.18 and Theorem 4.9
    \begin{enumerate}
      \item  There are 4 calls between $1 \mathrm{pm}$ and $3 \mathrm{pm}$, and 3 calls between $4 \mathrm{pm}$ and $4: 30 \mathrm{pm}$.
      \item  There are 6 calls between $2 \mathrm{pm}$ and $4 \mathrm{pm}$, but no calls between $3 \mathrm{pm}$ and $3: 30 \mathrm{pm}$.
      \item  There are 5 calls between $1 \mathrm{pm}$ and $3 \mathrm{pm}$, and 2 calls between $2 \mathrm{pm}$ and $4 \mathrm{pm}$.
    \end{enumerate}
\newpage

  \item Traffic on a particular road follows a Poisson process with rate $2 / 3$ 's of a vehicle per minute. 10 $\%$ of the vehicles are trucks, the other $90 \%$ are cars.

    See Exercise 4.21
    \begin{enumerate}
      \item  What is the probability that at least one truck passes in an hour?
      \item  What is the probability that out of the next three vehicles at least two are cars?
      \item  Find the conditional probability of seeing exactly 5 trucks and 45 cars in the next two hours given that 50 vehicles arrive in that time period.
      \item Find the conditional expectation of the number of vehicles it the next hour given that ten trucks arrive in that time period.
    \end{enumerate}
\newpage

  \item A lightbulb has a lifetime that is exponential with a mean of 200 days. When it burns out a janitor replaces it immediately. In addition there is a handyman who comes at times of a Poisson process at rate 0.01 day and replaces the bulb as "preventive maintenance."
    \begin{enumerate}
      \item  Consider the process of times when the bulb is replaced. What is this process?

        Let $X_1$ be the process of bulbs replaced by the janitor, and
        let $X_2$ be the process of bulbs replaced by the handyman.  And
        let $X = \operatorname{min}\{X_1, X_2\}$.

        Since $X_1$ is Exp we can convert it to Poisson with
        $\lambda = \frac{1}{200}$

        Then $X$ is the frequency of bulb replacement is given by
        \[ X \sim \operatorname{Pois}(.005 + .01) \]

      \item  What is the probability that the next two bulb replacements are done by the handyman?

        Let $\lambda$ be the rate for `janitor' and $\mu$ the rate
        for `handyman'.

        By superposition, the number of replacements by the handyman
        follow a Binomials distribution
        \[ \operatorname{Bin}\left(2,\frac{\lambda}{\lambda + \mu} \right) \]
        So P(within the next two replacements, both are by the handyman)
        \[ = \binom{2}{2} \left( \frac{\lambda}{\lambda + \mu} \right)^2  \]
      \item  In the long run what fraction of the replacements are due to failure?

        These are the bulbs replaced by the handyman
        \[ \frac{\lambda}{\lambda + \mu} \]


    \end{enumerate}
\end{enumerate}
%\newpage
%\section*{Theorems}
%
%\section*{Properties}
%%  \begin{enumerate}
%$  \end{enumerate}
%
\end{document}
