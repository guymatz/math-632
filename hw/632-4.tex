\documentclass{article} % This command is used to set the type of document you are working on such as an article, book, or presenation

%\usepackage[margin=1in]{geometry} % This package allows the editing of the page layout. I've set the margins to be 1 inch. 

\usepackage{amsmath, amsfonts}  % The first package allows the use of a large range of mathematical formula, commands, and symbols.  The second gives some useful mathematical fonts.

\usepackage{graphicx}  % This package allows the importing of images
\usepackage{marvosym}  % Lightning!

\usepackage{bbold}
%This allows us to use the theorem and proof environment 
\usepackage{enumitem}
\usepackage{amsthm}
\theoremstyle{plain}
\newtheorem*{theorem*}{Theorem}
\newtheorem{theorem}{Theorem}

\newtheoremstyle{case}{}{}{}{}{}{:}{ }{}
\theoremstyle{case}
\newtheorem{case}{Case}

%Custom commands.  
\newcommand{\abs}[1]{\left\lvert #1 \right\rvert} %absolute value command

%Custom symbols
\newcommand{\Rb}{\mathbb{R}}

\begin{document}

\begin{center}
\Large{\textbf{Assignment \#4}
            
UW-Madison MATH 632} % Name of course here
\vspace{5pt}
        
\normalsize{  Guy Matz% Your name here
        \\ Due: July 13, 2023}
\vspace{15pt}
\end{center}

\section*{Exercises}%

\begin{enumerate}[label={\fbox{\textbf{Exercise \#\arabic* :}}}]

  \item Familiarize yourself with the term undirected simple graph and the corresponding terms in Example 2.77 in the lecture notes. Consider a undirected simple graph with finite degrees. If $x \sim y$, we denote by $(x, y)=(y, x)$ the (undirected) edge that connects $x$ and $y$. Assume that $c$ is a positive function on the edges. That is, every edge $(x, y)$ is assigned a weight $c(x, y)=c(y, x)>0$ (called the conductance of the edge $(x, y)$ ).

A random walk on the graph with conductance $c$ is defined to be the Markov chain on the vertex set $\mathcal{V}$ with transition probability (for $x, y \in \mathcal{V}$ )

$$
p(x, y)= \begin{cases}\frac{c(x, y)}{\sum_{z: z \sim x} c(x, z)} & \text { if } x \sim y \\ 0 & \text { if } x \not\sim y .\end{cases}
$$

That is, when the random walker is at point $x$, it moves to one of its neighbor $y$ with a probability proportional to the conductance $c(x, y)$ of the edge $(x, y)$. You have seen an example of such a model in problem 8 of the practice exam.
  \begin{enumerate}
    \item  Verify that the measure $\mu$ defined by $\mu(x)=\sum_{z: z \sim x} c(x, z)$ is an invariant measure of the random walk. Is this measure reversible with respect to the Markov chain?

      min 10  and 35 of video 20230717

    \item  If the graph has only finitely many vertices, find an invariant distribution of the random walk (in terms of $\mu$ and $c$ ).
  \end{enumerate}


\newpage
  \item Suppose a taxicab driver moves between four locations $\mathbf{a}, \mathbf{b}, \mathbf{c}, \mathbf{d}$ following the Markov chain with transition matrix
\[
\end{array}\left(\begin{array}{cccc}
0 & .3 & 0 & .7 \\
.3 & 0 & .7 & 0 \\
.2 & .4 & .3 & .1 \\
.5 & .3 & 0 & .2
\end{array}\right)
\]
    \begin{enumerate}
      \item  What is the period of state a?

      \item  Compute its invariant distribution $\pi$.

      \item  Suppose the driver begins at location a at time 0, find the expectation of the number of times she visits location $\mathbf{c}$ before returning to $\mathbf{a}$.
    \end{enumerate}

\newpage
  \item  We model the daily weather of a town by a Markov chain $\left\{X_{n}: n \geq 0\right\}$ with state space $\mathcal{S}=\{S, C, R\}$ and transition matrix
\[
\end{array}\left(\begin{array}{ccc}
5/6 & 1/6 & 0 \\
1/6 & 1/3 & 1/2 \\
1/6 & 1/2 & 1/3 \\
\end{array}\right)
\]

  \begin{enumerate}
    \item  In the long run, what are the proportions of days that are sunny, cloudy, and rainy?

    \item  Given that today is cloudy, how many days does it take on average till the next cloudy day?

    \item  On average, a coffee store has 300, 210, and 140 customers on a sunny, cloudy, and rainy day, respectively. In the long run, what is the average number of customers per day?

    \item  Find the limits of $P_{S}\left(X_{n}=C\right), P_{S}\left(X_{n}=C, X_{n+2}=S\right)$, and $P_{S}\left(X_{n}=R, X_{2 n}=C\right)$ as $n \rightarrow \infty$.
  \end{enumerate}
\newpage
  \item Consider a Markov chain on the state space $\mathcal{S}=\{1,2,3,4,5\}$ with transition probability matrix
$$
\mathbf{P}=\left[\begin{array}{ccccc}
\frac{1}{4} & \frac{3}{4} & 0 & 0 & 0 \\
\frac{1}{2} & \frac{1}{2} & 0 & 0 & 0 \\
0 & 0 & \frac{1}{3} & \frac{2}{3} & 0 \\
0 & 0 & \frac{4}{5} & \frac{1}{5} & 0 \\
0 & \frac{1}{3} & 0 & \frac{1}{3} & \frac{1}{3}
\end{array}\right]
$$

  \begin{enumerate}
    \item  Find all stationary distributions of the MC.

      Hint: You may compare this problem to Example 2.62 in the book.

    \item  Find the limit of $\frac{1}{n} \sum_{k=1}^{n}\left|X_{k}-3\right|$ if we start the Markov chain from $X_{0}=2$.

    \item  Find the limit of $\frac{1}{n} \sum_{k=1}^{n}\left|X_{k}-3\right|$ if we start the Markov chain from uniform initial distribution. Hint: The answer may consist of different limiting values and their probabilities, like Example 2.90 in the lecture notes.

  \end{enumerate}

\newpage
  \item Let $p \in(0,1)$. Consider a Markov chain $\left(X_{n}\right)_{n \geq 0}$ on $\mathbb{Z}$ with transition probabilities $p(0,1)=$ $p(0,-1)=\frac{1}{2}$, and

    \[
      p(x, x+1)=1-p(x, x-1)= \begin{cases}p & \text { if } x>0 \\ 1-p & \text { if } x<0\end{cases}
  \]
  \begin{enumerate}
    \item  Find an invariant measure for this Markov chain. Hint: Consider the conductance $c(x, y)=$ $\left(\frac{p}{1-p}\right)^{\max \{|x|,|y|\}}$ on the edges, and use the conclusion in Problem 1(a).

    \item  For what values of $p \in(0,1)$ is this Markov chain recurrent? For what values of $p \in(0,1)$ is it positive recurrent? Hint: To answer the first question you may use the conclusions in Example 2.40 of the lecture notes. For the case $p>1 / 2$, it might be helpful to compare $\left|X_{n}\right|$ to the transient SRW in Example 2.40.

  \end{enumerate}



\end{enumerate*}

%\newpage
%\section*{Theorems}
%
%\section*{Properties}
%%  \begin{enumerate}
%$  \end{enumerate}
%
\end{document}
