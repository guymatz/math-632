\documentclass{article} % This command is used to set the type of document you are working on such as an article, book, or presenation

%\usepackage[margin=1in]{geometry} % This package allows the editing of the page layout. I've set the margins to be 1 inch. 

\usepackage{amsmath, amsfonts}  % The first package allows the use of a large range of mathematical formula, commands, and symbols.  The second gives some useful mathematical fonts.

\usepackage{graphicx}  % This package allows the importing of images
\usepackage{marvosym}  % Lightning!

%This allows us to use the theorem and proof environment 
\usepackage{enumitem}
\usepackage{amsthm}
\theoremstyle{plain}
\newtheorem*{theorem*}{Theorem}
\newtheorem{theorem}{Theorem}

\newtheoremstyle{case}{}{}{}{}{}{:}{ }{}
\theoremstyle{case}
\newtheorem{case}{Case}

%Custom commands.  
\newcommand{\abs}[1]{\left\lvert #1 \right\rvert} %absolute value command

%Custom symbols
\newcommand{\Rb}{\mathbb{R}}

\begin{document}

\begin{center}
\Large{\textbf{Assignment \#1}
            
UW-Madison MATH 632} % Name of course here
\vspace{5pt}
        
\normalsize{  Guy Matz% Your name here
        \\ Due: June 29, 2023}
\vspace{15pt}
\end{center}

\section*{Exercises}%
\begin{enumerate}[label={\fbox{\textbf{Exercise \#\arabic* :}}}]
  \item Monica works on a temporary basis. The mean length of each job she
    gets is 12 months and the amount of time she spends between jobs
    is exponential with mean 3 months.
    \begin{enumerate}
      \item In the long run what is the average number of jobs she works in a year?

\par\noindent\rule{\textwidth}{0.1pt}
        This is a Renewal Process wth $E[X] = 15 \text{ months } \cdot 
        \frac{1 \text{ year }}{12 \text{ months }} = \frac{5}{4} $ ,
        so the average number of jobs per year is given by
        \[ \lim_{t \to \infty} \frac{1}{E[X]}  = \frac{4}{5}  \]

      \item  In the long run what fraction of the time does she spend working?

\par\noindent\rule{\textwidth}{0.1pt}
        Let $L_n$ be the length of time at each job, where $E[L_n] = 
        \frac{12}{12}$, and let $W_n$ be the time off between jobs, with
        $E[L_n] = \frac{5}{12}$.
        Then this a Renewal-Reward Process with 
        \[ \lim_{t \to \infty} \frac{R_t}{t} =
        \frac{E[L_1]}{E[L_1] + E[W_1]}  = \frac{\frac{12}{12}}{\frac{12}{12} + \frac{3}{12} } = \frac{12}{15}    \]

    \end{enumerate}
  \newpage
  \item Three children take turns shooting a ball at a basket. They each
    shoot until they miss and then it is the next child’s turn. Suppose
    that child i $(1 \leq i \leq 3)$ makes a basket with probability $\pi$
    and that successive trials are independent. Determine the proportion of
    time in the long run that each child shoots. (Assume each shot takes
    one time unit.) \emph{Hint: Reviewing the geometric distribution
      could be helpful.}

\par\noindent\rule{\textwidth}{0.1pt}
      Let each child's attempt by Geomtric RV with $X_n \sim $ Geom($\pi$).
    so $E[X] = \frac{1}{\pi}$.
    And let $Y_n = X_1 + X_2 + X_3$, so $E[Y] = 3, E[X] = \frac{3}{\pi}$. 
    Then for this Renewal Reward cycle we get
    \[ \lim_{n \to \infty} \frac{R_t}{t} =
        \frac{\text{Rewards per cycle}}{\text{Cycle Length}}
        = \frac{E[X]}{E[Y]}
        = \frac{\pi}{3 \pi} = \frac{1}{3}   \]
  \newpage
  \item A device has two states: ON and OFF. When it works normally, it
    spends a random time in the ON state, then a random time in the OFF
    state, then again in the ON state, and so on. However, when switched
    from OFF to ON, the device fails with probability 0.01 and a
    replacement process is triggered immediately. Once a new device is in
    place, it is turned ON. Suppose that the average durations in the ON
    and OFF states are 40 and 20 minutes, respectively, and that it takes
    30 minutes to replace a failed device.
    \begin{enumerate}
      \item In the long run what is the rate at which devices are replaced?
\par\noindent\rule{\textwidth}{0.1pt}
        The mins per inter-arrival of $X$ is
        \[ X = 40 + 20 + 30 \cdot .01 = 60.3 \]
        $X \sim$ Geom(p), so 
        \[ E[X] = 60.3 \cdot \frac{1}{0.1} = 6030 \]

        Then
        \[ \lim_{t \to \infty} \frac{N_t}{t} = \frac{1}{E[X]} =  \frac{1}{6030} \]
      \item In the long run what is the fraction of time that the device is ON?
\par\noindent\rule{\textwidth}{0.1pt}

       The Expected value for a cycle is 60.3,
       so
       \[ \lim_{t \to \infty} \frac{R_t}{t} = \frac{40}{60.3} = .66 \]
    \end{enumerate}
  \newpage
  \item After dropping off a customer, an Uber driver needs on average 10
    minutes to receive a new ride request and pick up the customer.
    90\% of the customers are short distance riders who will pay
    \$10 on average and whose ride lasts on average 20 minutes.
    The remaining 10\% are long distance riders who pay \$25 on
    average and whose rides take 40 minutes on average.
    \begin{enumerate}
      \item In the long term, what is the rate of customers the Uber driver serves, in units of persons per hour?
\par\noindent\rule{\textwidth}{0.1pt}
        Let $X$ be the Variable for Ride Length, and $Y$ a variable for
        Ride Type, with a probability for a short ride, and long 
        defined as $P(Y=s) = 0.9, P(Y=l) = 0.1$, respectively.  Then
        the expected rate, in mimutes, is
       \begin{align*}
         E[X] = E[E[X|Y]] &= E[X|Y=s] \cdot P(Y=s) + E[X|Y=l] \cdot P(Y=l) \\
                          &= (20+10) \cdot 0.9 + (40+10) \cdot 0.1 = 32
       \end{align*}
       In hours, this is $32/60$ = .533

       Then the rate of customers the Uber driver serves, in units of persons per hour is
       \[ \lim_{t \to \infty} \frac{N_t}{t}  = \frac{1}{E[X]}  = \frac{1}{32/60} = 1.875 \]
     \item What is the long term rate of revenue (money coming in from the customers), in dollars per hour
\par\noindent\rule{\textwidth}{0.1pt}
        This is a Renewal-Reward process where the reward is the expected
        fare for one ride.  Let $F$ be the variable for the fare, then
       \begin{align*}
         E[F] = E[E[F|Y]] &=E[F|Y=s] \cdot P(Y=s) +E[F|Y=l] \cdot P(Y=l) \\
                          &= 10 \cdot 0.9 + 25 \cdot 0.1 = 11.5
       \end{align*}
        We find the long term rate of revenue, in dollars per hour, is
        \[ \lim_{t \to \infty} \frac{R_t}{t} = \frac{E[F]}{E[X]} = \frac{11.5}{.533} = 21.56 \]
    \end{enumerate}

  \newpage
  \item Let $\left\{Y_k\right\}_{k \geq 1}$ be i.i.d. $\mathbb{Z}_{\geq 0}$-valued random variables with p.m.f.
    \[
      P\left(Y_k=m\right)=2^{-1-m} \text { for } m \geq 0 .
      \]
    Let $S_0=0$ and $S_n=Y_1+\cdots+Y_n$ for $n \geq 1$
    \begin{enumerate}
      \item Is the process $\left\{Y_k\right\}_{k \geq 1}$ a Markov chain?
        Prove that it is not, or prove that it is and give its transition
        probability.
\par\noindent\rule{\textwidth}{0.1pt}
        We can see that the process is a Markov Chain by verifying that,
        for the $k+1$ term, conditioning on the $k{th}$ term is the same
        as conditioning on the term $\{1, \cdots, k\}$.   So we want to
        verify that:
        \[ P(Y_{k+1}= a_{k+1} | Y_{k} = a_k)=P(Y_{k+1}= a_{k+1} | Y_{k}=a_k, \dots , Y_{0}=a_0) \]

        And clearly it is, since all events $Y_k$ are i.i.d.  Due
        to the independence of events
        \[ P(Y_{k+1}=b | Y_{k} = a)=P(Y_{k+1}=b) = P(Y_1 = b)  \]
        The transition probability is given by the p.m.f:
        \[ P(Y_{n+1}=m) = 2^{-1-m} \]
      \item  $S_n$ is a Markov chain as can be seen readily for example by following the reasoning of Example 2.1. So you need not prove it. What are the suitable choices of state space for $S_n$? Find the two-step transition probability $p^{(2)}(x, y)$ for $S_n$.
\par\noindent\rule{\textwidth}{0.1pt}
        The State Space is $\Omega = \{ x \in \mathbb{N} \}$.

       The two-step transition probability $p^{(2)}(x, y)$ is equal to 
       the multiplication of two single-step transition functions using
       some intermediate point, e.g., from
       $x \rightarrow a \rightarrow y$, i.e.
       $p(x,a) \cdot p(a,y)$, which is
       \[ \sum^{y}_{a=x} 2^{-1-(a-x)} \cdot 2^{-1-(y-a)} = (y-x+1) \cdot 2^{-2 + (x-y)} \]
    \end{enumerate} 
\newpage
  \item A fair coin is tossed repeatedly with results $\left\{Y_k\right\}_{k \geq 0}$. That is, these random variables are i.i.d.  with common
    probability mass function
    \[ P(Y_k=0) = P(Y_k =1) = \frac{1}{2}  \]
    For $n \geq 1$ let $X_n = Y_{n-1} + Y_n$, the number of tails in the
    last two flips. Is Xn a Markov chain?
    Prove that it is or prove that it is not. An answer based on intuition
    alone may get some partial credit. Full credit comes from a rigorous
    solution. (Example 2.18 of the notes gives an example of how to prove
    that something is not a Markov chain.)
\par\noindent\rule{\textwidth}{0.1pt}
    The probability Mass Function for $X$ is 
    \[ P(X_n=0) = \frac{1}{4},  P(X_n=1) = \frac{1}{2},
          P(X_n=2)= \frac{1}{4}    \]
    From the construction of the process it follows that for any
    fixed $n \geq 1$ the random variables $X_n$ are NOT independent.
    Imagine the sequence of rolls
    \{\dots, $Y_{n-1}=0, Y_{n}=1, Y_{n+1}=0\}$, resulting in 
    $\{ \dots, X_{n-1}=1, X_{n}=1 \}$
    Then,
    \[ P(X_{n+1} = 2 | X_n = 1) = \frac{1}{2}  \]
    but
    \[ P(X_{n+1} = 2 | X_n = 1, X_{n-1} = 1) = 0  \]
    
\end{enumerate}

%\newpage
%\section*{Theorems}
%
%\section*{Properties}
%%  \begin{enumerate}
%$  \end{enumerate}
%
\end{document}
