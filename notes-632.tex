\documentclass{report}

%%%%%%%%%%%%%%%%%%%%%%%%%%%%%%%%%
% PACKAGE IMPORTS
%%%%%%%%%%%%%%%%%%%%%%%%%%%%%%%%%


\usepackage[tmargin=2cm,rmargin=1in,lmargin=1in,margin=0.85in,bmargin=2cm,footskip=.2in]{geometry}
\usepackage{amsmath,amsfonts,amsthm,amssymb,mathtools}
\usepackage[varbb]{newpxmath}
\usepackage{xfrac}
\usepackage[makeroom]{cancel}
\usepackage{mathtools}
\usepackage{bookmark}
\usepackage{enumitem}
\usepackage{hyperref,theoremref}
\hypersetup{
	pdftitle={Assignment},
	colorlinks=true, linkcolor=doc!90,
	bookmarksnumbered=true,
	bookmarksopen=true
}
\usepackage[most,many,breakable]{tcolorbox}
\usepackage{xcolor}
\usepackage{varwidth}
\usepackage{varwidth}
\usepackage{etoolbox}
%\usepackage{authblk}
\usepackage{nameref}
\usepackage{multicol,array}
\usepackage{tikz-cd}
\usepackage[ruled,vlined,linesnumbered]{algorithm2e}
\usepackage{comment} % enables the use of multi-line comments (\ifx \fi) 
\usepackage{import}
\usepackage{xifthen}
\usepackage{pdfpages}
\usepackage{transparent}

\newcommand\mycommfont[1]{\footnotesize\ttfamily\textcolor{blue}{#1}}
\SetCommentSty{mycommfont}
\newcommand{\incfig}[1]{%
    \def\svgwidth{\columnwidth}
    \import{./figures/}{#1.pdf_tex}
}

\usepackage{tikzsymbols}
\usepackage{tikz}
\usepackage{hhline}
\usepackage{float}
\renewcommand\qedsymbol{$\Laughey$}


%\usepackage{import}
%\usepackage{xifthen}
%\usepackage{pdfpages}
%\usepackage{transparent}


%%%%%%%%%%%%%%%%%%%%%%%%%%%%%%
% SELF MADE COLORS
%%%%%%%%%%%%%%%%%%%%%%%%%%%%%%



\definecolor{myg}{RGB}{56, 140, 70}
\definecolor{myb}{RGB}{45, 111, 177}
\definecolor{myr}{RGB}{199, 68, 64}
\definecolor{mytheorembg}{HTML}{F2F2F9}
\definecolor{mytheoremfr}{HTML}{00007B}
\definecolor{mylenmabg}{HTML}{FFFAF8}
\definecolor{mylenmafr}{HTML}{983b0f}
\definecolor{mypropbg}{HTML}{f2fbfc}
\definecolor{mypropfr}{HTML}{191971}
\definecolor{myexamplebg}{HTML}{F2FBF8}
\definecolor{myexamplefr}{HTML}{88D6D1}
\definecolor{myexampleti}{HTML}{2A7F7F}
\definecolor{mydefinitbg}{HTML}{E5E5FF}
\definecolor{mydefinitfr}{HTML}{3F3FA3}
\definecolor{notesgreen}{RGB}{0,162,0}
\definecolor{myp}{RGB}{197, 92, 212}
\definecolor{mygr}{HTML}{2C3338}
\definecolor{myred}{RGB}{127,0,0}
\definecolor{myyellow}{RGB}{169,121,69}
\definecolor{myexercisebg}{HTML}{F2FBF8}
\definecolor{myexercisefg}{HTML}{88D6D1}


%%%%%%%%%%%%%%%%%%%%%%%%%%%%
% TCOLORBOX SETUPS
%%%%%%%%%%%%%%%%%%%%%%%%%%%%

\setlength{\parindent}{1cm}
%================================
% THEOREM BOX
%================================

\tcbuselibrary{theorems,skins,hooks}
\newtcbtheorem[number within=section]{Theorem}{Theorem}
{%
	enhanced,
	breakable,
	colback = mytheorembg,
	frame hidden,
	boxrule = 0sp,
	borderline west = {2pt}{0pt}{mytheoremfr},
	sharp corners,
	detach title,
	before upper = \tcbtitle\par\smallskip,
	coltitle = mytheoremfr,
	fonttitle = \bfseries\sffamily,
	description font = \mdseries,
	separator sign none,
	segmentation style={solid, mytheoremfr},
}
{th}

\tcbuselibrary{theorems,skins,hooks}
\newtcbtheorem[number within=chapter]{theorem}{Theorem}
{%
	enhanced,
	breakable,
	colback = mytheorembg,
	frame hidden,
	boxrule = 0sp,
	borderline west = {2pt}{0pt}{mytheoremfr},
	sharp corners,
	detach title,
	before upper = \tcbtitle\par\smallskip,
	coltitle = mytheoremfr,
	fonttitle = \bfseries\sffamily,
	description font = \mdseries,
	separator sign none,
	segmentation style={solid, mytheoremfr},
}
{th}


\tcbuselibrary{theorems,skins,hooks}
\newtcolorbox{Theoremcon}
{%
	enhanced
	,breakable
	,colback = mytheorembg
	,frame hidden
	,boxrule = 0sp
	,borderline west = {2pt}{0pt}{mytheoremfr}
	,sharp corners
	,description font = \mdseries
	,separator sign none
}

%================================
% Corollery
%================================
\tcbuselibrary{theorems,skins,hooks}
\newtcbtheorem[number within=section]{Corollary}{Corollary}
{%
	enhanced
	,breakable
	,colback = myp!10
	,frame hidden
	,boxrule = 0sp
	,borderline west = {2pt}{0pt}{myp!85!black}
	,sharp corners
	,detach title
	,before upper = \tcbtitle\par\smallskip
	,coltitle = myp!85!black
	,fonttitle = \bfseries\sffamily
	,description font = \mdseries
	,separator sign none
	,segmentation style={solid, myp!85!black}
}
{th}
\tcbuselibrary{theorems,skins,hooks}
\newtcbtheorem[number within=chapter]{corollary}{Corollary}
{%
	enhanced
	,breakable
	,colback = myp!10
	,frame hidden
	,boxrule = 0sp
	,borderline west = {2pt}{0pt}{myp!85!black}
	,sharp corners
	,detach title
	,before upper = \tcbtitle\par\smallskip
	,coltitle = myp!85!black
	,fonttitle = \bfseries\sffamily
	,description font = \mdseries
	,separator sign none
	,segmentation style={solid, myp!85!black}
}
{th}


%================================
% LENMA
%================================

\tcbuselibrary{theorems,skins,hooks}
\newtcbtheorem[number within=section]{Lenma}{Lenma}
{%
	enhanced,
	breakable,
	colback = mylenmabg,
	frame hidden,
	boxrule = 0sp,
	borderline west = {2pt}{0pt}{mylenmafr},
	sharp corners,
	detach title,
	before upper = \tcbtitle\par\smallskip,
	coltitle = mylenmafr,
	fonttitle = \bfseries\sffamily,
	description font = \mdseries,
	separator sign none,
	segmentation style={solid, mylenmafr},
}
{th}

\tcbuselibrary{theorems,skins,hooks}
\newtcbtheorem[number within=chapter]{lenma}{Lenma}
{%
	enhanced,
	breakable,
	colback = mylenmabg,
	frame hidden,
	boxrule = 0sp,
	borderline west = {2pt}{0pt}{mylenmafr},
	sharp corners,
	detach title,
	before upper = \tcbtitle\par\smallskip,
	coltitle = mylenmafr,
	fonttitle = \bfseries\sffamily,
	description font = \mdseries,
	separator sign none,
	segmentation style={solid, mylenmafr},
}
{th}


%================================
% PROPOSITION
%================================

\tcbuselibrary{theorems,skins,hooks}
\newtcbtheorem[number within=section]{Prop}{Proposition}
{%
	enhanced,
	breakable,
	colback = mypropbg,
	frame hidden,
	boxrule = 0sp,
	borderline west = {2pt}{0pt}{mypropfr},
	sharp corners,
	detach title,
	before upper = \tcbtitle\par\smallskip,
	coltitle = mypropfr,
	fonttitle = \bfseries\sffamily,
	description font = \mdseries,
	separator sign none,
	segmentation style={solid, mypropfr},
}
{th}

\tcbuselibrary{theorems,skins,hooks}
\newtcbtheorem[number within=chapter]{prop}{Proposition}
{%
	enhanced,
	breakable,
	colback = mypropbg,
	frame hidden,
	boxrule = 0sp,
	borderline west = {2pt}{0pt}{mypropfr},
	sharp corners,
	detach title,
	before upper = \tcbtitle\par\smallskip,
	coltitle = mypropfr,
	fonttitle = \bfseries\sffamily,
	description font = \mdseries,
	separator sign none,
	segmentation style={solid, mypropfr},
}
{th}


%================================
% CLAIM
%================================

\tcbuselibrary{theorems,skins,hooks}
\newtcbtheorem[number within=section]{claim}{Claim}
{%
	enhanced
	,breakable
	,colback = myg!10
	,frame hidden
	,boxrule = 0sp
	,borderline west = {2pt}{0pt}{myg}
	,sharp corners
	,detach title
	,before upper = \tcbtitle\par\smallskip
	,coltitle = myg!85!black
	,fonttitle = \bfseries\sffamily
	,description font = \mdseries
	,separator sign none
	,segmentation style={solid, myg!85!black}
}
{th}



%================================
% Exercise
%================================

\tcbuselibrary{theorems,skins,hooks}
\newtcbtheorem[number within=section]{Exercise}{Exercise}
{%
	enhanced,
	breakable,
	colback = myexercisebg,
	frame hidden,
	boxrule = 0sp,
	borderline west = {2pt}{0pt}{myexercisefg},
	sharp corners,
	detach title,
	before upper = \tcbtitle\par\smallskip,
	coltitle = myexercisefg,
	fonttitle = \bfseries\sffamily,
	description font = \mdseries,
	separator sign none,
	segmentation style={solid, myexercisefg},
}
{th}

\tcbuselibrary{theorems,skins,hooks}
\newtcbtheorem[number within=chapter]{exercise}{Exercise}
{%
	enhanced,
	breakable,
	colback = myexercisebg,
	frame hidden,
	boxrule = 0sp,
	borderline west = {2pt}{0pt}{myexercisefg},
	sharp corners,
	detach title,
	before upper = \tcbtitle\par\smallskip,
	coltitle = myexercisefg,
	fonttitle = \bfseries\sffamily,
	description font = \mdseries,
	separator sign none,
	segmentation style={solid, myexercisefg},
}
{th}

%================================
% EXAMPLE BOX
%================================

\newtcbtheorem[number within=section]{Example}{Example}
{%
	colback = myexamplebg
	,breakable
	,colframe = myexamplefr
	,coltitle = myexampleti
	,boxrule = 1pt
	,sharp corners
	,detach title
	,before upper=\tcbtitle\par\smallskip
	,fonttitle = \bfseries
	,description font = \mdseries
	,separator sign none
	,description delimiters parenthesis
}
{ex}

\newtcbtheorem[number within=chapter]{example}{Example}
{%
	colback = myexamplebg
	,breakable
	,colframe = myexamplefr
	,coltitle = myexampleti
	,boxrule = 1pt
	,sharp corners
	,detach title
	,before upper=\tcbtitle\par\smallskip
	,fonttitle = \bfseries
	,description font = \mdseries
	,separator sign none
	,description delimiters parenthesis
}
{ex}

%================================
% DEFINITION BOX
%================================

\newtcbtheorem[number within=section]{Definition}{Definition}{enhanced,
	before skip=2mm,after skip=2mm, colback=red!5,colframe=red!80!black,boxrule=0.5mm,
	attach boxed title to top left={xshift=1cm,yshift*=1mm-\tcboxedtitleheight}, varwidth boxed title*=-3cm,
	boxed title style={frame code={
					\path[fill=tcbcolback]
					([yshift=-1mm,xshift=-1mm]frame.north west)
					arc[start angle=0,end angle=180,radius=1mm]
					([yshift=-1mm,xshift=1mm]frame.north east)
					arc[start angle=180,end angle=0,radius=1mm];
					\path[left color=tcbcolback!60!black,right color=tcbcolback!60!black,
						middle color=tcbcolback!80!black]
					([xshift=-2mm]frame.north west) -- ([xshift=2mm]frame.north east)
					[rounded corners=1mm]-- ([xshift=1mm,yshift=-1mm]frame.north east)
					-- (frame.south east) -- (frame.south west)
					-- ([xshift=-1mm,yshift=-1mm]frame.north west)
					[sharp corners]-- cycle;
				},interior engine=empty,
		},
	fonttitle=\bfseries,
	title={#2},#1}{def}
\newtcbtheorem[number within=chapter]{definition}{Definition}{enhanced,
	before skip=2mm,after skip=2mm, colback=red!5,colframe=red!80!black,boxrule=0.5mm,
	attach boxed title to top left={xshift=1cm,yshift*=1mm-\tcboxedtitleheight}, varwidth boxed title*=-3cm,
	boxed title style={frame code={
					\path[fill=tcbcolback]
					([yshift=-1mm,xshift=-1mm]frame.north west)
					arc[start angle=0,end angle=180,radius=1mm]
					([yshift=-1mm,xshift=1mm]frame.north east)
					arc[start angle=180,end angle=0,radius=1mm];
					\path[left color=tcbcolback!60!black,right color=tcbcolback!60!black,
						middle color=tcbcolback!80!black]
					([xshift=-2mm]frame.north west) -- ([xshift=2mm]frame.north east)
					[rounded corners=1mm]-- ([xshift=1mm,yshift=-1mm]frame.north east)
					-- (frame.south east) -- (frame.south west)
					-- ([xshift=-1mm,yshift=-1mm]frame.north west)
					[sharp corners]-- cycle;
				},interior engine=empty,
		},
	fonttitle=\bfseries,
	title={#2},#1}{def}



%================================
% Solution BOX
%================================

\makeatletter
\newtcbtheorem{question}{Question}{enhanced,
	breakable,
	colback=white,
	colframe=myb!80!black,
	attach boxed title to top left={yshift*=-\tcboxedtitleheight},
	fonttitle=\bfseries,
	title={#2},
	boxed title size=title,
	boxed title style={%
			sharp corners,
			rounded corners=northwest,
			colback=tcbcolframe,
			boxrule=0pt,
		},
	underlay boxed title={%
			\path[fill=tcbcolframe] (title.south west)--(title.south east)
			to[out=0, in=180] ([xshift=5mm]title.east)--
			(title.center-|frame.east)
			[rounded corners=\kvtcb@arc] |-
			(frame.north) -| cycle;
		},
	#1
}{def}
\makeatother

%================================
% SOLUTION BOX
%================================

\makeatletter
\newtcolorbox{solution}{enhanced,
	breakable,
	colback=white,
	colframe=myg!80!black,
	attach boxed title to top left={yshift*=-\tcboxedtitleheight},
	title=Solution,
	boxed title size=title,
	boxed title style={%
			sharp corners,
			rounded corners=northwest,
			colback=tcbcolframe,
			boxrule=0pt,
		},
	underlay boxed title={%
			\path[fill=tcbcolframe] (title.south west)--(title.south east)
			to[out=0, in=180] ([xshift=5mm]title.east)--
			(title.center-|frame.east)
			[rounded corners=\kvtcb@arc] |-
			(frame.north) -| cycle;
		},
}
\makeatother

%================================
% Question BOX
%================================

\makeatletter
\newtcbtheorem{qstion}{Question}{enhanced,
	breakable,
	colback=white,
	colframe=mygr,
	attach boxed title to top left={yshift*=-\tcboxedtitleheight},
	fonttitle=\bfseries,
	title={#2},
	boxed title size=title,
	boxed title style={%
			sharp corners,
			rounded corners=northwest,
			colback=tcbcolframe,
			boxrule=0pt,
		},
	underlay boxed title={%
			\path[fill=tcbcolframe] (title.south west)--(title.south east)
			to[out=0, in=180] ([xshift=5mm]title.east)--
			(title.center-|frame.east)
			[rounded corners=\kvtcb@arc] |-
			(frame.north) -| cycle;
		},
	#1
}{def}
\makeatother

\newtcbtheorem[number within=chapter]{wconc}{Wrong Concept}{
	breakable,
	enhanced,
	colback=white,
	colframe=myr,
	arc=0pt,
	outer arc=0pt,
	fonttitle=\bfseries\sffamily\large,
	colbacktitle=myr,
	attach boxed title to top left={},
	boxed title style={
			enhanced,
			skin=enhancedfirst jigsaw,
			arc=3pt,
			bottom=0pt,
			interior style={fill=myr}
		},
	#1
}{def}



%================================
% NOTE BOX
%================================

%\usetikzlibrary{automata,arrows,positioning,calc,shadows,blur}
\usetikzlibrary{automata,arrows,positioning,calc,shadows}

\tcbuselibrary{skins}
\newtcolorbox{note}[1][]{%
	enhanced jigsaw,
	colback=gray!20!white,%
	colframe=gray!80!black,
	size=small,
	boxrule=1pt,
	title=\textbf{Note:-},
	halign title=flush center,
	coltitle=black,
	breakable,
	drop shadow=black!50!white,
	attach boxed title to top left={xshift=1cm,yshift=-\tcboxedtitleheight/2,yshifttext=-\tcboxedtitleheight/2},
	minipage boxed title=1.5cm,
	boxed title style={%
			colback=white,
			size=fbox,
			boxrule=1pt,
			boxsep=2pt,
			underlay={%
					\coordinate (dotA) at ($(interior.west) + (-0.5pt,0)$);
					\coordinate (dotB) at ($(interior.east) + (0.5pt,0)$);
					\begin{scope}
						\clip (interior.north west) rectangle ([xshift=3ex]interior.east);
						%\filldraw [white, blur shadow={shadow opacity=60, shadow yshift=-.75ex}, rounded corners=2pt] (interior.north west) rectangle (interior.south east);
						\filldraw [white, rounded corners=2pt] (interior.north west) rectangle (interior.south east);
					\end{scope}
					\begin{scope}[gray!80!black]
						\fill (dotA) circle (2pt);
						\fill (dotB) circle (2pt);
					\end{scope}
				},
		},
	#1,
}

%%%%%%%%%%%%%%%%%%%%%%%%%%%%%%
% SELF MADE COMMANDS
%%%%%%%%%%%%%%%%%%%%%%%%%%%%%%


\newcommand{\thm}[2]{\begin{Theorem}{#1}{}#2\end{Theorem}}
\newcommand{\cor}[2]{\begin{Corollary}{#1}{}#2\end{Corollary}}
\newcommand{\mlenma}[2]{\begin{Lenma}{#1}{}#2\end{Lenma}}
\newcommand{\mprop}[2]{\begin{Prop}{#1}{}#2\end{Prop}}
\newcommand{\clm}[3]{\begin{claim}{#1}{#2}#3\end{claim}}
\newcommand{\wc}[2]{\begin{wconc}{#1}{}\setlength{\parindent}{1cm}#2\end{wconc}}
\newcommand{\thmcon}[1]{\begin{Theoremcon}{#1}\end{Theoremcon}}
\newcommand{\ex}[2]{\begin{Example}{#1}{}#2\end{Example}}
\newcommand{\dfn}[2]{\begin{Definition}[colbacktitle=red!75!black]{#1}{}#2\end{Definition}}
\newcommand{\dfnc}[2]{\begin{definition}[colbacktitle=red!75!black]{#1}{}#2\end{definition}}
\newcommand{\qs}[2]{\begin{question}{#1}{}#2\end{question}}
\newcommand{\pf}[2]{\begin{myproof}[#1]#2\end{myproof}}
\newcommand{\nt}[1]{\begin{note}#1\end{note}}

\newcommand*\circled[1]{\tikz[baseline=(char.base)]{
		\node[shape=circle,draw,inner sep=1pt] (char) {#1};}}
\newcommand\getcurrentref[1]{%
	\ifnumequal{\value{#1}}{0}
	{??}
	{\the\value{#1}}%
}
\newcommand{\getCurrentSectionNumber}{\getcurrentref{section}}
\newenvironment{myproof}[1][\proofname]{%
	\proof[\bfseries #1: ]%
}{\endproof}

\newcommand{\mclm}[2]{\begin{myclaim}[#1]#2\end{myclaim}}
\newenvironment{myclaim}[1][\claimname]{\proof[\bfseries #1: ]}{}

\newcounter{mylabelcounter}

\makeatletter
\newcommand{\setword}[2]{%
	\phantomsection
	#1\def\@currentlabel{\unexpanded{#1}}\label{#2}%
}
\makeatother




\tikzset{
	symbol/.style={
			draw=none,
			every to/.append style={
					edge node={node [sloped, allow upside down, auto=false]{$#1$}}}
		}
}


% deliminators
\DeclarePairedDelimiter{\abs}{\lvert}{\rvert}
\DeclarePairedDelimiter{\norm}{\lVert}{\rVert}

\DeclarePairedDelimiter{\ceil}{\lceil}{\rceil}
\DeclarePairedDelimiter{\floor}{\lfloor}{\rfloor}
\DeclarePairedDelimiter{\round}{\lfloor}{\rceil}

\newsavebox\diffdbox
\newcommand{\slantedromand}{{\mathpalette\makesl{d}}}
\newcommand{\makesl}[2]{%
\begingroup
\sbox{\diffdbox}{$\mathsurround=0pt#1\mathrm{#2}$}%
\pdfsave
\pdfsetmatrix{1 0 0.2 1}%
\rlap{\usebox{\diffdbox}}%
\pdfrestore
\hskip\wd\diffdbox
\endgroup
}
\newcommand{\dd}[1][]{\ensuremath{\mathop{}\!\ifstrempty{#1}{%
\slantedromand\@ifnextchar^{\hspace{0.2ex}}{\hspace{0.1ex}}}%
{\slantedromand\hspace{0.2ex}^{#1}}}}
\ProvideDocumentCommand\dv{o m g}{%
  \ensuremath{%
    \IfValueTF{#3}{%
      \IfNoValueTF{#1}{%
        \frac{\dd #2}{\dd #3}%
      }{%
        \frac{\dd^{#1} #2}{\dd #3^{#1}}%
      }%
    }{%
      \IfNoValueTF{#1}{%
        \frac{\dd}{\dd #2}%
      }{%
        \frac{\dd^{#1}}{\dd #2^{#1}}%
      }%
    }%
  }%
}
\providecommand*{\pdv}[3][]{\frac{\partial^{#1}#2}{\partial#3^{#1}}}
%  - others
\DeclareMathOperator{\Lap}{\mathcal{L}}
\DeclareMathOperator{\Var}{Var} % varience
\DeclareMathOperator{\Cov}{Cov} % covarience
\DeclareMathOperator{\E}{E} % expected

% Since the amsthm package isn't loaded

% I prefer the slanted \leq
\let\oldleq\leq % save them in case they're every wanted
\let\oldgeq\geq
\renewcommand{\leq}{\leqslant}
\renewcommand{\geq}{\geqslant}

% % redefine matrix env to allow for alignment, use r as default
% \renewcommand*\env@matrix[1][r]{\hskip -\arraycolsep
%     \let\@ifnextchar\new@ifnextchar
%     \array{*\c@MaxMatrixCols #1}}


%\usepackage{framed}
%\usepackage{titletoc}
%\usepackage{etoolbox}
%\usepackage{lmodern}


%\patchcmd{\tableofcontents}{\contentsname}{\sffamily\contentsname}{}{}

%\renewenvironment{leftbar}
%{\def\FrameCommand{\hspace{6em}%
%		{\color{myyellow}\vrule width 2pt depth 6pt}\hspace{1em}}%
%	\MakeFramed{\parshape 1 0cm \dimexpr\textwidth-6em\relax\FrameRestore}\vskip2pt%
%}
%{\endMakeFramed}

%\titlecontents{chapter}
%[0em]{\vspace*{2\baselineskip}}
%{\parbox{4.5em}{%
%		\hfill\Huge\sffamily\bfseries\color{myred}\thecontentspage}%
%	\vspace*{-2.3\baselineskip}\leftbar\textsc{\small\chaptername~\thecontentslabel}\\\sffamily}
%{}{\endleftbar}
%\titlecontents{section}
%[8.4em]
%{\sffamily\contentslabel{3em}}{}{}
%{\hspace{0.5em}\nobreak\itshape\color{myred}\contentspage}
%\titlecontents{subsection}
%[8.4em]
%{\sffamily\contentslabel{3em}}{}{}  
%{\hspace{0.5em}\nobreak\itshape\color{myred}\contentspage}



%%%%%%%%%%%%%%%%%%%%%%%%%%%%%%%%%%%%%%%%%%%
% TABLE OF CONTENTS
%%%%%%%%%%%%%%%%%%%%%%%%%%%%%%%%%%%%%%%%%%%

\usepackage{tikz}
\definecolor{doc}{RGB}{0,60,110}
\usepackage{titletoc}
\contentsmargin{0cm}
\titlecontents{chapter}[3.7pc]
{\addvspace{30pt}%
	\begin{tikzpicture}[remember picture, overlay]%
		\draw[fill=doc!60,draw=doc!60] (-7,-.1) rectangle (-0.9,.5);%
		\pgftext[left,x=-3.5cm,y=0.2cm]{\color{white}\Large\sc\bfseries Chapter\ \thecontentslabel};%
	\end{tikzpicture}\color{doc!60}\large\sc\bfseries}%
{}
{}
{\;\titlerule\;\large\sc\bfseries Page \thecontentspage
	\begin{tikzpicture}[remember picture, overlay]
		\draw[fill=doc!60,draw=doc!60] (2pt,0) rectangle (4,0.1pt);
	\end{tikzpicture}}%
\titlecontents{section}[3.7pc]
{\addvspace{2pt}}
{\contentslabel[\thecontentslabel]{2pc}}
{}
{\hfill\small \thecontentspage}
[]
\titlecontents*{subsection}[3.7pc]
{\addvspace{-1pt}\small}
{}
{}
{\ --- \small\thecontentspage}
[ \textbullet\ ][]

\makeatletter
\renewcommand{\tableofcontents}{%
	\chapter*{%
	  \vspace*{-20\p@}%
	  \begin{tikzpicture}[remember picture, overlay]%
		  \pgftext[right,x=15cm,y=0.2cm]{\color{doc!60}\Huge\sc\bfseries \contentsname};%
		  \draw[fill=doc!60,draw=doc!60] (13,-.75) rectangle (20,1);%
		  \clip (13,-.75) rectangle (20,1);
		  \pgftext[right,x=15cm,y=0.2cm]{\color{white}\Huge\sc\bfseries \contentsname};%
	  \end{tikzpicture}}%
	\@starttoc{toc}}
\makeatother


%From M275 "Topology" at SJSU
\newcommand{\id}{\mathrm{id}}
\newcommand{\taking}[1]{\xrightarrow{#1}}
\newcommand{\inv}{^{-1}}

%From M170 "Introduction to Graph Theory" at SJSU
\DeclareMathOperator{\diam}{diam}
\DeclareMathOperator{\ord}{ord}
\newcommand{\defeq}{\overset{\mathrm{def}}{=}}

%From the USAMO .tex files
\newcommand{\ts}{\textsuperscript}
\newcommand{\dg}{^\circ}
\newcommand{\ii}{\item}

% % From Math 55 and Math 145 at Harvard
% \newenvironment{subproof}[1][Proof]{%
% \begin{proof}[#1] \renewcommand{\qedsymbol}{$\blacksquare$}}%
% {\end{proof}}

\newcommand{\liff}{\leftrightarrow}
\newcommand{\lthen}{\rightarrow}
\newcommand{\opname}{\operatorname}
\newcommand{\surjto}{\twoheadrightarrow}
\newcommand{\injto}{\hookrightarrow}
\newcommand{\On}{\mathrm{On}} % ordinals
\DeclareMathOperator{\img}{im} % Image
\DeclareMathOperator{\Img}{Im} % Image
\DeclareMathOperator{\coker}{coker} % Cokernel
\DeclareMathOperator{\Coker}{Coker} % Cokernel
\DeclareMathOperator{\Ker}{Ker} % Kernel
\DeclareMathOperator{\rank}{rank}
\DeclareMathOperator{\Spec}{Spec} % spectrum
\DeclareMathOperator{\Tr}{Tr} % trace
\DeclareMathOperator{\pr}{pr} % projection
\DeclareMathOperator{\ext}{ext} % extension
\DeclareMathOperator{\pred}{pred} % predecessor
\DeclareMathOperator{\dom}{dom} % domain
\DeclareMathOperator{\ran}{ran} % range
\DeclareMathOperator{\Hom}{Hom} % homomorphism
\DeclareMathOperator{\Mor}{Mor} % morphisms
\DeclareMathOperator{\End}{End} % endomorphism

\newcommand{\eps}{\epsilon}
\newcommand{\veps}{\varepsilon}
\newcommand{\ol}{\overline}
\newcommand{\ul}{\underline}
\newcommand{\wt}{\widetilde}
\newcommand{\wh}{\widehat}
\newcommand{\vocab}[1]{\textbf{\color{blue} #1}}
\providecommand{\half}{\frac{1}{2}}
\newcommand{\dang}{\measuredangle} %% Directed angle
\newcommand{\ray}[1]{\overrightarrow{#1}}
\newcommand{\seg}[1]{\overline{#1}}
\newcommand{\arc}[1]{\wideparen{#1}}
\DeclareMathOperator{\cis}{cis}
\DeclareMathOperator*{\lcm}{lcm}
\DeclareMathOperator*{\argmin}{arg min}
\DeclareMathOperator*{\argmax}{arg max}
\newcommand{\cycsum}{\sum_{\mathrm{cyc}}}
\newcommand{\symsum}{\sum_{\mathrm{sym}}}
\newcommand{\cycprod}{\prod_{\mathrm{cyc}}}
\newcommand{\symprod}{\prod_{\mathrm{sym}}}
\newcommand{\Qed}{\begin{flushright}\qed\end{flushright}}
\newcommand{\parinn}{\setlength{\parindent}{1cm}}
\newcommand{\parinf}{\setlength{\parindent}{0cm}}
% \newcommand{\norm}{\|\cdot\|}
\newcommand{\inorm}{\norm_{\infty}}
\newcommand{\opensets}{\{V_{\alpha}\}_{\alpha\in I}}
\newcommand{\oset}{V_{\alpha}}
\newcommand{\opset}[1]{V_{\alpha_{#1}}}
\newcommand{\lub}{\text{lub}}
\newcommand{\del}[2]{\frac{\partial #1}{\partial #2}}
\newcommand{\Del}[3]{\frac{\partial^{#1} #2}{\partial^{#1} #3}}
\newcommand{\deld}[2]{\dfrac{\partial #1}{\partial #2}}
\newcommand{\Deld}[3]{\dfrac{\partial^{#1} #2}{\partial^{#1} #3}}
\newcommand{\lm}{\lambda}
\newcommand{\uin}{\mathbin{\rotatebox[origin=c]{90}{$\in$}}}
\newcommand{\usubset}{\mathbin{\rotatebox[origin=c]{90}{$\subset$}}}
\newcommand{\lt}{\left}
\newcommand{\rt}{\right}
\newcommand{\bs}[1]{\boldsymbol{#1}}
\newcommand{\exs}{\exists}
\newcommand{\st}{\strut}
\newcommand{\dps}[1]{\displaystyle{#1}}

\newcommand{\sol}{\setlength{\parindent}{0cm}\textbf{\textit{Solution:}}\setlength{\parindent}{1cm} }
\newcommand{\solve}[1]{\setlength{\parindent}{0cm}\textbf{\textit{Solution: }}\setlength{\parindent}{1cm}#1 \Qed}

% Things Lie
\newcommand{\kb}{\mathfrak b}
\newcommand{\kg}{\mathfrak g}
\newcommand{\kh}{\mathfrak h}
\newcommand{\kn}{\mathfrak n}
\newcommand{\ku}{\mathfrak u}
\newcommand{\kz}{\mathfrak z}
\DeclareMathOperator{\Ext}{Ext} % Ext functor
\DeclareMathOperator{\Tor}{Tor} % Tor functor
\newcommand{\gl}{\opname{\mathfrak{gl}}} % frak gl group
\renewcommand{\sl}{\opname{\mathfrak{sl}}} % frak sl group chktex 6

% More script letters etc.
\newcommand{\SA}{\mathcal A}
\newcommand{\SB}{\mathcal B}
\newcommand{\SC}{\mathcal C}
\newcommand{\SF}{\mathcal F}
\newcommand{\SG}{\mathcal G}
\newcommand{\SH}{\mathcal H}
\newcommand{\OO}{\mathcal O}

\newcommand{\SCA}{\mathscr A}
\newcommand{\SCB}{\mathscr B}
\newcommand{\SCC}{\mathscr C}
\newcommand{\SCD}{\mathscr D}
\newcommand{\SCE}{\mathscr E}
\newcommand{\SCF}{\mathscr F}
\newcommand{\SCG}{\mathscr G}
\newcommand{\SCH}{\mathscr H}

% Mathfrak primes
\newcommand{\km}{\mathfrak m}
\newcommand{\kp}{\mathfrak p}
\newcommand{\kq}{\mathfrak q}

% number sets
\newcommand{\RR}[1][]{\ensuremath{\ifstrempty{#1}{\mathbb{R}}{\mathbb{R}^{#1}}}}
\newcommand{\NN}[1][]{\ensuremath{\ifstrempty{#1}{\mathbb{N}}{\mathbb{N}^{#1}}}}
\newcommand{\ZZ}[1][]{\ensuremath{\ifstrempty{#1}{\mathbb{Z}}{\mathbb{Z}^{#1}}}}
\newcommand{\QQ}[1][]{\ensuremath{\ifstrempty{#1}{\mathbb{Q}}{\mathbb{Q}^{#1}}}}
\newcommand{\CC}[1][]{\ensuremath{\ifstrempty{#1}{\mathbb{C}}{\mathbb{C}^{#1}}}}
\newcommand{\PP}[1][]{\ensuremath{\ifstrempty{#1}{\mathbb{P}}{\mathbb{P}^{#1}}}}
\newcommand{\HH}[1][]{\ensuremath{\ifstrempty{#1}{\mathbb{H}}{\mathbb{H}^{#1}}}}
\newcommand{\FF}[1][]{\ensuremath{\ifstrempty{#1}{\mathbb{F}}{\mathbb{F}^{#1}}}}
% expected value
\newcommand{\EE}{\ensuremath{\mathbb{E}}}
\newcommand{\charin}{\text{ char }}
\DeclareMathOperator{\sign}{sign}
\DeclareMathOperator{\Aut}{Aut}
\DeclareMathOperator{\Inn}{Inn}
\DeclareMathOperator{\Syl}{Syl}
\DeclareMathOperator{\Gal}{Gal}
\DeclareMathOperator{\GL}{GL} % General linear group
\DeclareMathOperator{\SL}{SL} % Special linear group

%---------------------------------------
% BlackBoard Math Fonts :-
%---------------------------------------

%Captital Letters
\newcommand{\bbA}{\mathbb{A}}	\newcommand{\bbB}{\mathbb{B}}
\newcommand{\bbC}{\mathbb{C}}	\newcommand{\bbD}{\mathbb{D}}
\newcommand{\bbE}{\mathbb{E}}	\newcommand{\bbF}{\mathbb{F}}
\newcommand{\bbG}{\mathbb{G}}	\newcommand{\bbH}{\mathbb{H}}
\newcommand{\bbI}{\mathbb{I}}	\newcommand{\bbJ}{\mathbb{J}}
\newcommand{\bbK}{\mathbb{K}}	\newcommand{\bbL}{\mathbb{L}}
\newcommand{\bbM}{\mathbb{M}}	\newcommand{\bbN}{\mathbb{N}}
\newcommand{\bbO}{\mathbb{O}}	\newcommand{\bbP}{\mathbb{P}}
\newcommand{\bbQ}{\mathbb{Q}}	\newcommand{\bbR}{\mathbb{R}}
\newcommand{\bbS}{\mathbb{S}}	\newcommand{\bbT}{\mathbb{T}}
\newcommand{\bbU}{\mathbb{U}}	\newcommand{\bbV}{\mathbb{V}}
\newcommand{\bbW}{\mathbb{W}}	\newcommand{\bbX}{\mathbb{X}}
\newcommand{\bbY}{\mathbb{Y}}	\newcommand{\bbZ}{\mathbb{Z}}

%---------------------------------------
% MathCal Fonts :-
%---------------------------------------

%Captital Letters
\newcommand{\mcA}{\mathcal{A}}	\newcommand{\mcB}{\mathcal{B}}
\newcommand{\mcC}{\mathcal{C}}	\newcommand{\mcD}{\mathcal{D}}
\newcommand{\mcE}{\mathcal{E}}	\newcommand{\mcF}{\mathcal{F}}
\newcommand{\mcG}{\mathcal{G}}	\newcommand{\mcH}{\mathcal{H}}
\newcommand{\mcI}{\mathcal{I}}	\newcommand{\mcJ}{\mathcal{J}}
\newcommand{\mcK}{\mathcal{K}}	\newcommand{\mcL}{\mathcal{L}}
\newcommand{\mcM}{\mathcal{M}}	\newcommand{\mcN}{\mathcal{N}}
\newcommand{\mcO}{\mathcal{O}}	\newcommand{\mcP}{\mathcal{P}}
\newcommand{\mcQ}{\mathcal{Q}}	\newcommand{\mcR}{\mathcal{R}}
\newcommand{\mcS}{\mathcal{S}}	\newcommand{\mcT}{\mathcal{T}}
\newcommand{\mcU}{\mathcal{U}}	\newcommand{\mcV}{\mathcal{V}}
\newcommand{\mcW}{\mathcal{W}}	\newcommand{\mcX}{\mathcal{X}}
\newcommand{\mcY}{\mathcal{Y}}	\newcommand{\mcZ}{\mathcal{Z}}


%---------------------------------------
% Bold Math Fonts :-
%---------------------------------------

%Captital Letters
\newcommand{\bmA}{\boldsymbol{A}}	\newcommand{\bmB}{\boldsymbol{B}}
\newcommand{\bmC}{\boldsymbol{C}}	\newcommand{\bmD}{\boldsymbol{D}}
\newcommand{\bmE}{\boldsymbol{E}}	\newcommand{\bmF}{\boldsymbol{F}}
\newcommand{\bmG}{\boldsymbol{G}}	\newcommand{\bmH}{\boldsymbol{H}}
\newcommand{\bmI}{\boldsymbol{I}}	\newcommand{\bmJ}{\boldsymbol{J}}
\newcommand{\bmK}{\boldsymbol{K}}	\newcommand{\bmL}{\boldsymbol{L}}
\newcommand{\bmM}{\boldsymbol{M}}	\newcommand{\bmN}{\boldsymbol{N}}
\newcommand{\bmO}{\boldsymbol{O}}	\newcommand{\bmP}{\boldsymbol{P}}
\newcommand{\bmQ}{\boldsymbol{Q}}	\newcommand{\bmR}{\boldsymbol{R}}
\newcommand{\bmS}{\boldsymbol{S}}	\newcommand{\bmT}{\boldsymbol{T}}
\newcommand{\bmU}{\boldsymbol{U}}	\newcommand{\bmV}{\boldsymbol{V}}
\newcommand{\bmW}{\boldsymbol{W}}	\newcommand{\bmX}{\boldsymbol{X}}
\newcommand{\bmY}{\boldsymbol{Y}}	\newcommand{\bmZ}{\boldsymbol{Z}}
%Small Letters
\newcommand{\bma}{\boldsymbol{a}}	\newcommand{\bmb}{\boldsymbol{b}}
\newcommand{\bmc}{\boldsymbol{c}}	\newcommand{\bmd}{\boldsymbol{d}}
\newcommand{\bme}{\boldsymbol{e}}	\newcommand{\bmf}{\boldsymbol{f}}
\newcommand{\bmg}{\boldsymbol{g}}	\newcommand{\bmh}{\boldsymbol{h}}
\newcommand{\bmi}{\boldsymbol{i}}	\newcommand{\bmj}{\boldsymbol{j}}
\newcommand{\bmk}{\boldsymbol{k}}	\newcommand{\bml}{\boldsymbol{l}}
\newcommand{\bmm}{\boldsymbol{m}}	\newcommand{\bmn}{\boldsymbol{n}}
\newcommand{\bmo}{\boldsymbol{o}}	\newcommand{\bmp}{\boldsymbol{p}}
\newcommand{\bmq}{\boldsymbol{q}}	\newcommand{\bmr}{\boldsymbol{r}}
\newcommand{\bms}{\boldsymbol{s}}	\newcommand{\bmt}{\boldsymbol{t}}
\newcommand{\bmu}{\boldsymbol{u}}	\newcommand{\bmv}{\boldsymbol{v}}
\newcommand{\bmw}{\boldsymbol{w}}	\newcommand{\bmx}{\boldsymbol{x}}
\newcommand{\bmy}{\boldsymbol{y}}	\newcommand{\bmz}{\boldsymbol{z}}

%---------------------------------------
% Scr Math Fonts :-
%---------------------------------------

\newcommand{\sA}{{\mathscr{A}}}   \newcommand{\sB}{{\mathscr{B}}}
\newcommand{\sC}{{\mathscr{C}}}   \newcommand{\sD}{{\mathscr{D}}}
\newcommand{\sE}{{\mathscr{E}}}   \newcommand{\sF}{{\mathscr{F}}}
\newcommand{\sG}{{\mathscr{G}}}   \newcommand{\sH}{{\mathscr{H}}}
\newcommand{\sI}{{\mathscr{I}}}   \newcommand{\sJ}{{\mathscr{J}}}
\newcommand{\sK}{{\mathscr{K}}}   \newcommand{\sL}{{\mathscr{L}}}
\newcommand{\sM}{{\mathscr{M}}}   \newcommand{\sN}{{\mathscr{N}}}
\newcommand{\sO}{{\mathscr{O}}}   \newcommand{\sP}{{\mathscr{P}}}
\newcommand{\sQ}{{\mathscr{Q}}}   \newcommand{\sR}{{\mathscr{R}}}
\newcommand{\sS}{{\mathscr{S}}}   \newcommand{\sT}{{\mathscr{T}}}
\newcommand{\sU}{{\mathscr{U}}}   \newcommand{\sV}{{\mathscr{V}}}
\newcommand{\sW}{{\mathscr{W}}}   \newcommand{\sX}{{\mathscr{X}}}
\newcommand{\sY}{{\mathscr{Y}}}   \newcommand{\sZ}{{\mathscr{Z}}}


%---------------------------------------
% Math Fraktur Font
%---------------------------------------

%Captital Letters
\newcommand{\mfA}{\mathfrak{A}}	\newcommand{\mfB}{\mathfrak{B}}
\newcommand{\mfC}{\mathfrak{C}}	\newcommand{\mfD}{\mathfrak{D}}
\newcommand{\mfE}{\mathfrak{E}}	\newcommand{\mfF}{\mathfrak{F}}
\newcommand{\mfG}{\mathfrak{G}}	\newcommand{\mfH}{\mathfrak{H}}
\newcommand{\mfI}{\mathfrak{I}}	\newcommand{\mfJ}{\mathfrak{J}}
\newcommand{\mfK}{\mathfrak{K}}	\newcommand{\mfL}{\mathfrak{L}}
\newcommand{\mfM}{\mathfrak{M}}	\newcommand{\mfN}{\mathfrak{N}}
\newcommand{\mfO}{\mathfrak{O}}	\newcommand{\mfP}{\mathfrak{P}}
\newcommand{\mfQ}{\mathfrak{Q}}	\newcommand{\mfR}{\mathfrak{R}}
\newcommand{\mfS}{\mathfrak{S}}	\newcommand{\mfT}{\mathfrak{T}}
\newcommand{\mfU}{\mathfrak{U}}	\newcommand{\mfV}{\mathfrak{V}}
\newcommand{\mfW}{\mathfrak{W}}	\newcommand{\mfX}{\mathfrak{X}}
\newcommand{\mfY}{\mathfrak{Y}}	\newcommand{\mfZ}{\mathfrak{Z}}
%Small Letters
\newcommand{\mfa}{\mathfrak{a}}	\newcommand{\mfb}{\mathfrak{b}}
\newcommand{\mfc}{\mathfrak{c}}	\newcommand{\mfd}{\mathfrak{d}}
\newcommand{\mfe}{\mathfrak{e}}	\newcommand{\mff}{\mathfrak{f}}
\newcommand{\mfg}{\mathfrak{g}}	\newcommand{\mfh}{\mathfrak{h}}
\newcommand{\mfi}{\mathfrak{i}}	\newcommand{\mfj}{\mathfrak{j}}
\newcommand{\mfk}{\mathfrak{k}}	\newcommand{\mfl}{\mathfrak{l}}
\newcommand{\mfm}{\mathfrak{m}}	\newcommand{\mfn}{\mathfrak{n}}
\newcommand{\mfo}{\mathfrak{o}}	\newcommand{\mfp}{\mathfrak{p}}
\newcommand{\mfq}{\mathfrak{q}}	\newcommand{\mfr}{\mathfrak{r}}
\newcommand{\mfs}{\mathfrak{s}}	\newcommand{\mft}{\mathfrak{t}}
\newcommand{\mfu}{\mathfrak{u}}	\newcommand{\mfv}{\mathfrak{v}}
\newcommand{\mfw}{\mathfrak{w}}	\newcommand{\mfx}{\mathfrak{x}}
\newcommand{\mfy}{\mathfrak{y}}	\newcommand{\mfz}{\mathfrak{z}}


\title{
  \Huge{Math 431 - Introduction to Probability Theory}
  \\
  Notes
}
\author{\huge{Guy Matz}}
\date{}

\begin{document}
%\maketitle

% \setcounter{chapter}{1}
\chapter*{20230620 - Renewal Process}

\begin{itemize}
  \item Renewal Process - Video I @ min 59
    \begin{itemize}
      \item $N_t$: Number of arrivals before time t.  It is indexed by
        time, so it is a continuous-time stochastic process
      \item $S_n$: Arrival time of $n^{th}$ "particle"
      \item $X_n$: Inter-arrival time.  Time between $n$ and $n-1$ (iid)
    \end{itemize}
  \item Definitions
    \dfn{ Arrival Time }{
      Let $\{X_i: i = 1,2...\}$ be a sequence o iid strictly positive
      RVs.  Let $S_0 = 0$,
        \[ S_n = X_1 + \dots X_n, n \geq 1 \]
      Then the process
      \[ N_t \text{max} \{ n \geq 0; S_n \leq t \} \]
      is called the Renewal Process corresponding to $\{X_i; i \geq 1 \}$
    }

    \dfn{ Renewal Process }{
      Let $\{X_i : i=1,2...\{$ be a sequence of i.i.d. (See video
          around min 12
          \[ \lim_{t \to \infty} \frac{N_t}{t} = \frac{1}{E[X_1]}   \]
    }

    \dfn{ stochastic process }{
      A series of random variables indexed by time
    }

    \dfn{ IID Process }{
      Independent, Identically distributed RVs
    }

    \dfn{ Renewal-Reward Process}{
      Let $\{N_t: t \geq 0 \}  $ be a renewal process with strictly 
      positive inter-arrival times $\{X_k : k \geq 1 \}$.  Let 
      $\{ Y_n : n \geq 1 \}$ be iid random variables, i.e. "rewards".
      Then the process
      \[ R_t = \sum^{N_t}_{n=1} Y_n, t \geq 0 \]
      is called the Renewal-Reward Process corresponding to 
      inter-arrival times $(X_k)_{k=1}^{\infty}$ and rewards
      $(Y_n)_{n=1}^{\infty}$
    }
    \nt{
      Renewal-Reward Process, whenever a renewal happens, we get a reward

      Generalization of a Renewal process, where the reward is always 1
    }

    \thm{ SLLN for RRP }{
    \[ \lim_{t \to \infty} \frac{R_t}{t}  =  \frac{E[Y_1]}{E[X_1]}  \]
    }

    \item Properties of a Renewal Processes
      \begin{enumerate}
        \item $N_0 = 0$
        \item $N_t$ is increasing in time, and right-continuous
        \item $\mathcal{Z}_{\geq 0}$-valued piece-wise function
        \item $\lim_{t \to \infty} N_t = \infty$
      \end{enumerate}

      \thm{SLLN (Strong Law of Large Numbers) for RP (Renewal Process)} {
          \[ \lim_{t \to \infty} \frac{N_t}{t} = \frac{1}{E[X_1]} \]
      }
      \item Examples: See times (min) in video
        \ex{VIDEO 2, min 28: The Lifeftime of a lightbulb
          $\sim Exp(\frac{1}{100})$. When
        a lightbulb burns out, it takes 1 hour for the technician to 
        notice.  It is then immediately replace.  What is the long-term
        rate off consumption of lightbulbs?}
        {
          Let $L_k$ be the lifeftime of the k-th bulb.
          Then the inter-arrival time is $X_k = L_k + 1$.  $N_t$ is the
          number of bulbs replaced before time t.  By SLLN for RP,
          \[ \lim_{t \to \infty} \frac{N_t}{t} = \frac{1}{E[X_1]}  
          = \frac{1}{E[L1 + 1]} = \frac{1}{100 + 1} \]
          So the long-term replacement rate is 1 bulb every 101 days.
        }

        \ex{VIDEO 2, min 38: Same as above, but the technician only checks
          the bulb exactly on the hour}
        {
          Let $L_k$ be the lifeftime of the k-th bulb.
          Then the inter-arrival time is now an integer, since the
          replacement happens on the hour. 
          $X_k = \lceil L_k \rceil$.  $N_t$ is the
          number of bulbs replaced before time t.  By SLLN for RP,
          \[ \lim_{t \to \infty} \frac{N_t}{t} = \frac{1}{E[X_1]}   \]
          \[ P(X=m) = P(m-1 < L \leq m) = P(L_1 > m-1) - P(L > m)
        = e^{-\frac{1}{100}(m-1)} - e^{-\frac{1}{100}m} \]
        \[ = p^{m-1}(1-p) \]
        I.e. $X_1 \sim Geom(1-e^{-\frac{1}{100}})$.
        So
        \[ E[X_1] = \frac{1}{1 - e^{-\frac{1}{100}}}  \]
        So the long-term replacement rate is 1 over that!
        }

        \ex{VIDEO 2, min 49: \underline{Renewal-Reward Process} - 
          Bus arrives at a station with inter-arrival time of 10 mins, on
          avergae.  The number of customers, on average, get off.  What
          is the long-term rate of passengers that get off at the station?
        }
        {
          Intuitive answer.  3 passengers every 10 mins, or 18 / hr.

          $X_k$ is the interarrival time of the buses. $Y_k$ is the 
          number of passengers that get off the bus (reward).  Let $R_t$
          be the number of people who get off the bus before time t.
          \[ R_t = \sum^{N_t}_{k=1} Y_k \]
          The longterm rate of passengers is
          \begin{align*}
            \lim_{t \to \infty} \frac{R_t}{t} &= \lim_{t \to \infty} \frac{1}{t} \sum^{N_t}_{k=1} Y_k \\
                  &= \lim_{t \to \infty} \frac{N_t}{t} \frac{1}{N_t} 
                             \sum^{N_t}_{k=1} Y_k \\
                  &= \frac{E[Y_1]}{E[X_1]} \\
                  &= \frac{3}{10} 
          \end{align*}
        }
        
        \nt{
          \[ \lim_{t \to \infty} \frac{R_t}{t} = \frac{E[Y_1]}{E[X_1]}  \]
          Where $E[Y_1]$ is the Reward mean, and $E[X_1]$ is the 
          cycle length.

          \textbf{So the long-term ratio of rewards is the mean reward
          per cycle, divided by the mean of cycle length}
        }

      \item Questions
        \begin{itemize}
          \item So, $X_n$ is a time for nth particle,  $S_n$ is total time,
            and $N_t$ is \# of arrivals before time t, an integer
          \item What is the difference between $S_{N_t}$ and $t$?
        \end{itemize}
      

  \end{itemize}

\chapter*{20230621 - Renewal-Reward / Markov Chains}
\section*{Renewal Reward}%

  \ex{On-OFF Process} {
    See min 23 of video 3

    \textit{A machine alternates between "ON" and "OFF".  On avergae it
    spends 5 hours at "ON" state, and 3 hours at "OFFF".  What is the
  limiting proportion of time spent in the "ON" state?}

    We say "a renewal happens" when it is switched from "OF" to "ON"

    Let $O_k, F_k$ denote the time spend in ON, OFF states during the k-th
    renewal cycle.
    \[ X_k = O_k + _k \]

    Consider the time ON as a reward

    \[ \sum^{N_t}_{k=1} O_k \leq R_t \leq \sum^{N_t+1}_{k=1} O_k \]
    ??? Because we're not counting OFF for $R_t$ ???
    \[ \frac{1}{t} \sum^{N_t}_{k=1} O_k \leq frac{R_t}{t} \leq \frac{1}{t} \sum^{N_t+1}_{k=1} O_k \]

   \[ \frac{N_t}{t} \frac{1}{N_t} \sum^{N_t}_{k=1} O_k \leq frac{R_t}{t} \leq \frac{N_t+1}{t} \frac{1}{N_t+1} \sum^{N_t+1}_{k=1} O_k \]
    \[ \frac{1}{E[X_1]} E[O_1]  \]

    See video at min 35 for more

  }

  Take-away message:  The SLLN  for RRP still works even though rewards
  do not come as "packages"

  \section*{Markov Chains : Video 0621 @ min 43}%
  \dfn{ Markov Chain }{
    A discrete-time stochastic process $\{X_t: k \in \mathbb{Z}_{\geq 0}\}$
      with countable state space $S$ is said to be a Markov Chain if
      \begin{align*}
         &P(X_{n+1}=a_{n+1}|X_n=a_n, x_{n-1}=a_{n-1},\dots,X_0=a_0) \\
       = &P(X_{n+1}=a_{n+1}|X_n=a_n)
      \end{align*}

      for all $a_0, a_1, \dots, a_{n+1} \in S, n \geq 0$

      If $P(X_{n+1}=a_n+1|X_n=a_n)$ does not depend in n, then the Markov Chain is said to be "time-homogeneous" (because IID) , and
        \[ P(a,b) := P(X_{n+1}=b|X_n =a ) \]
        is the \underline{transition probability} of jumping rfom a to b.
  }
  \ex{HOW TO CHECK THE MARKOV Property - video 0621, min 51} {
    Every IID process $\{X_k: k \geq 0\}$ is a time-homogeneous MC.
    \[ P(X_{n+1} = a_{n+1} | X_n=a_n, \dots , x_0 = a_0) = 
    P(X_{n+1}=a_{n+1}) \]

    and,
    \[ P(X_{n+1} = a_{n+1} | X_n=a_n) = P(X_{n+1}=a_{n+1}) \]

    So to check MC, verify that 
    \[ P(X_{n+1} = a_{n+1} | X_n=a_n, \dots , x_0 = a_0) = 
     P(X_{n+1} = a_{n+1} | X_n=a_n) \]
   Hence the Markov Property holds and $(X_k)_{k \geq 0}$ is a Markov Chain

   Moreover, because $X$ are iid,
   \[ P(X_{n+1} = b | X_n = a) = P(X_{n+1}=b) =  P(X_1=b) \]
   does not depend on $n$, so the Markov Chain is time-homogeneous, with
   \[ p(a,b) = P(X_1 = b) \]
  }
  
%  \ex{SRW: Video 0621 - min 58} {
%    The Simple Random Walk (SRW) is also a MC.
%
%    Let $(Y_k: k \geq 1)$ be iid RVs with $P(Y_1 = 1) = p$ and
%    $P(Y_1 = -1) = (1-p)$
%
%    Let $S_0 = 0, S_n = \sum^{\infty}_{k=1} , n \geq 1$
%
%    To check the Markov Property, we need to verify the conditioning on all
%    states equals the conditioning on the current state, I.e
%    \[ P(S_{n+1}=a_{n+1} | S_n=a_n,  \dots, S_0 = 0)
%      =  \frac{P(S_{n+1}=a_{n+1} , S_n=a_n,  \dots, S_0 = 0)}
%      {P(S_n=a_n,  \dots, S_0 = 0)}
%      \[  = \frac{P(Y_1=a_1,Y_2=a_2-a_1, \dots Y_{n+1}=a_{n+1}-a_n}
%        {P(Y_1=a_1,Y_2=a_2-a_1, \dots Y_{n}=a_{n}-a_{n-1}}
%      \]
%  }
  \ex{For us to show: @ min 1:09} {
    Let $\{Y_k, k\geq 0\}$ be iid RVs.  Let $X_k=(Y_k-1, Y_k), k \geq 1$.  
    Then $\{X_n, k \leq 1\}$ is a MC.  Show this!
  }

\chapter*{20230622 - Markov Chains}
  \begin{itemize}
    \item Conditioning on the entire history is the same as conditioning
      on the current state
    \dfn{ Markov Chain }{
      Discrete-time Stochastic Process
    }
    \dfn{ Time-Homogeneous }{
      Transition probability does not depend on time.  I.e, if
      \[ P(X_{n+1} = b | n_n=a) = p(a,b) \]
      does not depend on time
    }
    \item Examples
      \item SRW
      \item IID process
    \dfn{ Transition Probability }{
      P of transition from a to b, i.e. $p(a,b)$
    }
    \dfn{ Transition Probability Function }{
      A function $p : S \times S \rightarrow [0,1]$ is called a transition
      probability function if
        \[ \sum^{}_{b \in S} p(a,b) = 1 \]
      for all $a \in S$
    }
    \nt{
      For a time-homogensous MC, $P(X_{n+1}=b| X_n=a) = p(a,b)$
        deffines a transition prob function
    }
    \ex{a Video 20230622} {
      \begin{itemize}
        \item rolling dice @ min 17?

        \item (Biased) SRW @ min 18?
          \begin{align}
            P(a,b) = P(S_{n+1}=b|S_n=a) &= 0 \text{ if } |b-a| > 1 \\
                                       &= p  \text{ if } b=a + 1 \\
                                       &= 1-p  \text{ if } b=a - 
          \end{align}
        \item IID (min 24)

          Let $Y_0, Y_1, \dots$ be iid with prob
          \[ P(Y_0 = 1) = P(Y_0 = -1) = \frac{1}{2} \]
          Then $X_n = (Y_n, Y_{n-1}) , n \geq 1$ is a MC with state space
          \[ S =  \{ (-1, -1), (-1, 1), (1, -1), (1, 1) \} = \{-1,1\}^2\]
          Its' Transition Probabiity Function
          \begin{align}
            P( (a_1, a_2), (b_1, b_2) &= P(X_{n+1} = (b_1, b_2) | X_n = (a_1, a_2)) \\
                                  &= P(Y_{n+1} = b_1, Y_n=b_2| Y_n = a_1, Y_{n-1}=a_2)
                                  &= \frac{P(Y_{n+1} = b_1, Y_n=b_2, Y_n = a_1, Y_{n-1}=a_2)}{P(Y_n=a_1, Y_{n-1} = a_2)}
                                  &= 0 \text{ if } a_1 \neq b_2
                                  &= \frac{1}{2}  \text{ if } a_1 = b_2
          \end{align}
          For all $(a_1, a_2), (b_1, b_2) \in S$
        \item Gambler's Ruin at min 35

          Win with prob p (lose with prob 1-p).  The games stops when
          fortune is 0 or \$5.  $X_0 = x$.  Transition probability function
          \begin{align}
            p(a,b) &= p \text{ if } b = a+1, 1 \leq a \leq 4 \\
                   &= 1-p \text{ if } b = a-1, 1 \leq a \leq 4 \\
                   &= 1 \text{ if } a=b \\
                   &= 0 \text{ otherwise }
          \end{align}

          \dfn{ Absorbing State}{
            If a state $a \in S$ such that $p(a,a) = 1$ , then a
            is called an absorbing state
          }
          \dfn{ Initial Distribution }{
            Let $\mu$ be a prob dist on $S$.  We say that the MC
            $(X_n), n \geq 0$ has initial dist $\mu$ is $P(X_0=x)=\mu(x)
            , x \in S$
          }
          \thm{ title } {
            For any prod dist $\mu$ on $S$ and trans prob func
            $p(\cdot,\cdot)$ there exists a time-homo MC with state space
            $S$ , init dist $\mu$ and trans prob $p(\cdot,\cdot)$
          }
          \ex{@ 1:08} {
            example of Gambler's ruin by MC
          }
      \end{itemize}
      \nt{
        If a MC has initial distribution $\mu$, then it's distribution
        is denoted by $P_{\mu}$
      }
    }
  \end{itemize}

\chapter*{20230626 - Markov Chains}
  \begin{itemize}
    \item  Success Run Chain: success move you forward, failure moves
      back to beginning
    \item $P_a = P_{\delta_a}$: The dist of a MC with initial dist
      $\mu = \delta_a$, i.e. $X_0 = a$ with prob 1
    \item The Expectation with respect to $P_{\mu}$ and $P_a$ are
      denoted by $E_{\mu}$ and $E_a$, respectively
    \item Let $X_0$ be a RV .  Let $(Y_k)_{k \in \mathbb{N}}$ be a seq
      of indep. RVs.  Let $g$ be a certain function.

      Then the RVs $(X_n)_{n \in \mathbb{Z} \geq 0}$ defined by 
      \[ X_{n+1} = g(X_n, Y_{n+1}) \]
      is a Markov Chain

  \end{itemize}

  \section*{Examples}%
    \begin{itemize}
    \item min 22 - Transition Probability Function
    \item min 45 - Success-Run chain with $g$ function above
    \item min 110 - Multi-step MC
    \end{itemize}
  
\chapter{Definitions}%
  \dfn{Stopping Time, p. 39}{
    The RV, $T$  with values $\mathbb{Z}_{\geq0} \cup \{ \infty \}$ is a 
    \underline{Stopping Time} (for the process $X$) if, for each
    nonnegative imteger $n$, there is a subset $C_n \subset S^{n+1}$
    such that
    \[ \{ T = n \} = \{(X_0, \dots , X_n) \in C \} \]
    I.e., for each $n$, the values $(X_0, \dots , X_n)$ determine whether
    $T=n$ happens or not.

    I.E! The first recurrence of some particular thing
  }
\chapter*{20230627 - Markov Chains}
   \begin{itemize}
     \item $E_\mu[f(x)] = \mu P^n f(x) $
     \item Markov into the infinite future
         \[ P_\mu\left( (x_{n+1}, \dots) \in A | X_n=x,(X_k, \dots, X_n) \in B \right) = P_x 
         \left( ( X_{n+1}, \dots) \in A) \right) \]
   \end{itemize}
   \section*{String Markov Property}%
   \dfn{ Stopping Time }{
     A random variable $T \in [0, \infty]$ is said to be a \underline{Stooping Time} if for any $n \geq 0$
     there exists  a set $C_n$ such that
     \[ \{T = n \} = \{ (X_0, \dots , X_n) \in C_n \} \]
     When $C_n$ denotes the set of sequences $(x_1, \dots x_n)$ where $x_1 \neq x, \dots, x_{n-1} \neq x$
     and $x_n = x$
     \ex{$\{T_x = n\}$} {
      \[\{T_x = n\} = \{ X_1 \neq x, X_2 \neq x, \dots, X_n = x\} \]
       So, $X_n = x$
     }
   }
   \nt{
     To verify a \underline{Stopping Time},  justify that $\{T=n\}$ only
     involves the history of the MC up to time n.
   }
   
   \dfn{ Hitting Time }{
     For any set $A \subset S$, we denote the \underline{hitting time} by
     \[ \tau_A = inf \{n \geq 0: X_n \in A \} \]
     Then $\tau$ is also a \underline{Stopping Time} because
     \[ \{ \tau_A = n \} = \{ X_0 \notin A, X_{n-1} \notin A, X_n \in A \} = \{ (X_0, \dots, X_n) \in C_n \} \]
   }
   \dfn{ Strong Markov Peroprty }{
     For a time-homogeneous MC and and set A,B and any $\mu$, and any stopping time T
     \[ P_\mu (  (X_{T+1}, \dots ) \in A | X_T=x, T > \infty , (X_0, \dots, X_T) \in B = P_x( (X_1, X_2 \dots ) \in A) \]
     The probability that the future behaves like A, is the same as a MC starting fresh
   }
   
\chapter*{20230628 - Markov Chains}
  \nt{
    Notation:
    \begin{itemize}
      \item Stopping time, $T $is the $n$ where $X_n = 3$
        \[ T = inf \{n \geq0: X_n = 3\} \]
      \item The probability that at 1 past the Stopping Time $T$, $X_n = 2$, given that the
        value of $X_n$ at the Stopping Time is 3, and the Stopping Time is finite
        \[ P(X_{T+1}=2 | X_T=3, T < \infty) \]
      \item $\rho_{xy}: $ Probability that starting from x, we will visit y in finite time
        \[ \rho_{xy} = P_x(X_n = y \text{ for some  } n \geq 1) \]
      \item Time atfer $k^{th}$ return to x that you return to x, i.e. the k-th return time to x
        \[ T^{k+1}_x = inf \{n > T_x^k : X_n = x \}  \]
      \item k-th return to y
        \[ T^k_y \]
      \item Probability of the k-th return to y after starting from x ?
        \[ P_x(T^k_y) \]
      \item $N_x$ : Total number of visits to x (after time 1)
      \item $E_x[N_y]$: Starting from x, how many times will we visit y
          \[ E_x[N_y] = \sum^{\infty}_{n=1} P(X_n=y) = \sum^{\infty}_{n=1} T_y^k < \infty  \]
          \[ E_x[N_y] = \sum^{\infty}_{k=1} \rho_{xy} \cdot \rho^{k-1}_{yy}   \]
            See min 57 off 20230628 video
      \item $R_x$ is called the "Communicating Class of x"
      \item $T^k_x$: Time to return to x, k times
        \[ P_x(\{T^k_x < \infty \text{ for all } k \geq 1\}) = 0 \]
        Means that, starting at x, the probabity is 0 that the time to return to x k-times is finite

    \end{itemize}
  }
  \section*{Transience nad Recurrence}%
  \dfn{ Transience }{
    If $\rho_{xx} < 1$
    \[ P_x(T_x^k = \infty \text{ for some k } = 1 \]
    \[ P_x(T_x^k < \infty \text{ for all k } = 0 \]
    $N_x < \infty$
    To prove that a state is transient, show that it has a P > 0 of escaping frfom the state, and there
    is a recurrent state with p=1
  }
  \dfn{ Recurrence }{
    If $\rho_{xx} = 1$
    \[ P_x(T^k_x < \infty, \text{ for all  } k \geq 1) = 1 \]
    \[ P_x(T^k_x = \infty, \text{ for all  } k \geq 1) = 0 \]
      \[N_x = \infty \]
      Recurrence is contagious, i.e. any state that communicates with a recurrent state is recurrent
  }
  \[ 
  E_x[N_y] =
  \begin{cases}
    0 &\text{ if } \rho_{xy} = 0 \\
    \infty &\text{ if } \rho_{xy} > 0 \& y \text{ is recurrent } \\
    \frac{p_{xy}}{1 - \rho_{xy}}  &\text{ if } \rho_{xy} > 0 \& y \text{ is transient }
  \end{cases}
\]
  \thm{ $x$ is recurrent iff } {
    $x$ is recurrent $\Leftrightarrow E_x[N_x] = \infty \Leftrightarrow N_x = \infty
    \Leftrightarrow \\sum^{\infty}_{n=1} p^n(x,y) = \infty$
  }
  \myproof {
    The return time to x is > 0, to $T_0 \sim $ Geom(1-p) implies
    $p(0,0) = p_0(T_0 < \infty) = 1$
  }

    \ex{SSRW - Symmwetric Simple Random Walk} {
      SSRW is recurrent, i.e. all states are recurrent..  We only need to show that "0" is recurrent. I.e
      \begin{align}
        \infty = E_0[N_0] &= \sum^{\infty}_{n=1} p^n(0,0) \\
                          &=  \sum^{\infty}_{k=1} p^{2k}(0,0) \\
                          &=  \sum^{\infty}_{k=1} \binom{2k}{k} \frac{1}{2} ^{2k} \\
                          &=  \sum^{\infty}_{k=1} \frac{2k!}{k!k!} \frac{1}{2} ^{2k} \\ \\
                          &=  \text{ Stirling }
      \end{align}
    }
    \section*{ Classification of States}
    \nt{
      \begin{itemize}
        \item $x \rightarrow y$: if $p^n(x,y) > 0$ for some $n \geq 0$ and we say y is accessible from x
        \item $x \leftrightarrow y$: if $x \rightarrow y$ and $x \leftarrow y$, we say x and y communicate
      \end{itemize}
    }
    \thm{ title } {
      \begin{itemize}
        \item if $x \neq y$, $x \rightarrow y$, and $\rho_{yx} < 1$, then x is transient
        \item if $x \neq y$, $x$ is recurrent and $x rightarrow y$, then $\rho_{yx} = 1$, and y is recrrent
      \end{itemize}
    }
\chapter*{20230629 - Markov Chains}
    \dfn{ Irreducible }{
      $A \subset S$ is irredicible if $x \leftrightarrow y$ for all $xx,y \in A$

      We say that $B \subset S$ is closed if for any $x \in B, y \notin B$, we have $x \not\rightarrow y$ 

      $R_x$ is called the "Communicating Class of x"
    }
    `
  \thm{ title } {
    Every finite closed subsets of a state space contains at least one recurrent state

    So every irredicible finite-stae MC is recurrent
  }


\chapter*{20230703 - Markov Chains}
  \dfn{ Canonical Decomposition }{
    Union of transient state  with the union of all recurrent states

    See min 10 of 20230703
  }

  Every finite-time Markov chain has an  - eventually - absorbing state

  \[ m(x) = E_x[\tau] = (I-Q)^{-1} 1_{x} \]
    Where 1 is a column matrix of "1"s.  This is the expected time
    spent in transient states

    $u(x,y)$ is the probability of going from a specific transient
    state to a specific recurrent state. Probabilityof being absorbed

    \section{Invariant Measure}%
    Describes the long-term behaviour , long-term frequency to 
    a state

    

\chapter*{Notation}%
  \begin{itemize}
      \item $N_t$: Number of arrivals before time t.  It is indexed by
        time, so it is a continuous-time stochastic process
      \item $S_n$: Arrival time of $n^{th}$ "particle"
      \item $\rho_{xy}: $ Probability that starting from x, we will visit y in finite time
        \[ \rho_{xy} = P_x(X_n = y \text{ for some  } n \geq 1) \]
      \item Time atfer $k^{th}$ return to x that you return to x, i.e. the k-th return time to x
        \[ T^{k+1}_x = inf \{n > T_x^k : X_n = x \}  \]
      \item k-th return to y
        \[ T^k_y \]
      \item Probability of the k-th return to y after starting from x ?
        \[ P_x(T^k_y) \]
      \item $N_x$ : Total number of visits to x (after time 1)
      \item $E_x[N_y]$: Starting from x, how many times will we visit y
          \[ E_x[N_y] = \sum^{\infty}_{n=1} P(X_n=y) = \sum^{\infty}_{n=1} T_y^k < \infty  \]
          \[ E_x[N_y] = \sum^{\infty}_{k=1} \rho_{xy} \cdot \rho^{k-1}_{yy}   \]
            See min 57 off 20230628 video
      \item $R_x$ is called the "Communicating Class of x"
      \item $T^k_x$: Time to return to x, k times
        \[ P_x(\{T^k_x < \infty \text{ for all } k \geq 1\}) = 0 \]
        Means that, starting at x, the probabity is 0 that the time to return to x k-times is finite
  \end{itemize}

\chapter*{Definitions}%
    \dfn{ Arrival Time }{
      Let $\{X_i: i = 1,2...\}$ be a sequence o iid strictly positive
      RVs.  Let $S_0 = 0$,
        \[ S_n = X_1 + \dots X_n, n \geq 1 \]
      Then the process
      \[ N_t \text{max} \{ n \geq 0; S_n \leq t \} \]
      is called the Renewal Process corresponding to $\{X_i; i \geq 1 \}$
    }
    \dfn{ Renewal Process }{
      Let $\{X_i : i=1,2...\{$ be a sequence of i.i.d. (See video
          around min 12
          \[ \lim_{t \to \infty} \frac{N_t}{t} = \frac{1}{E[X_1]}   \]
    }

    \dfn{ stochastic process }{
      A series of random variables indexed by time
    }

    \dfn{ IID Process }{
      Independent, Identically distributed RVs
    }

    \dfn{ Renewal-Reward Process}{
      Let $\{N_t: t \geq 0 \}  $ be a renewal process with strictly 
      positive inter-arrival times $\{X_k : k \geq 1 \}$.  Let 
      $\{ Y_n : n \geq 1 \}$ be iid random variables, i.e. "rewards".
      Then the process
      \[ R_t = \sum^{N_t}_{n=1} Y_n, t \geq 0 \]
      is called the Renewal-Reward Process corresponding to 
      inter-arrival times $(X_k)_{k=1}^{\infty}$ and rewards
      $(Y_n)_{n=1}^{\infty}$

      \thm{ SLLN for RRP }{
      \[\lim_{t \to \infty} \frac{R_t}{t}  =  \frac{E[Y_1]}{E[X_1]}  \]
      }
    }
  \dfn{ Markov Chain }{
    A discrete-time stochastic process $\{X_t: k \in \mathbb{Z}_{\geq 0}\}$
      with countable state space $S$ is said to be a Markov Chain if
      \begin{align*}
         &P(X_{n+1}=a_{n+1}|X_n=a_n, x_{n-1}=a_{n-1},\dots,X_0=a_0) \\
       = &P(X_{n+1}=a_{n+1}|X_n=a_n)
      \end{align*}

      for all $a_0, a_1, \dots, a_{n+1} \in S, n \geq 0$

      If $P(X_{n+1}=a_n+1|X_n=a_n)$ does not depend in n, then the Markov Chain is said to be "time-homogeneous" (because IID) , and
        \[ P(a,b) := P(X_{n+1}=b|X_n =a ) \]
        is the \underline{transition probability} of jumping rfom a to b.
  }
    \dfn{ Time-Homogeneous }{
      Transition probability does not depend on time.  I.e, if
      \[ P(X_{n+1} = b | n_n=a) = p(a,b) \]
      does not depend on time
    }
    \dfn{ Transition Probability }{
      P of transition from a to b, i.e. $p(a,b)$
    }
    \dfn{ Transition Probability Function }{
      A function $p : S \times S \rightarrow [0,1]$ is called a transition
      probability function if
        \[ \sum^{}_{b \in S} p(a,b) = 1 \]
      for all $a \in S$
    }
     \dfn{ Absorbing State}{
       If a state $a \in S$ such that $p(a,a) = 1$ , then a
       is called an absorbing state
     }
     \dfn{ Initial Distribution }{
            Let $\mu$ be a prob dist on $S$.  We say that the MC
            $(X_n), n \geq 0$ has initial dist $\mu$ is $P(X_0=x)=\mu(x)
            , x \in S$
     }
  \dfn{Stopping Time, p. 39}{
    The RV, $T$  with values $\mathbb{Z}_{\geq0} \cup \{ \infty \}$ is a 
    \underline{Stopping Time} (for the process $X$) if, for each
    nonnegative imteger $n$, there is a subset $C_n \subset S^{n+1}$
    such that
    \[ \{ T = n \} = \{(X_0, \dots , X_n) \in C \} \]
    I.e., for each $n$, the values $(X_0, \dots , X_n)$ determine whether
    $T=n$ happens or not.

    I.E! The first recurrence of some particular thing
      E.g Stopping time, $T $is the $n$ where $X_n = 3$ for
        \[ T = inf \{n \geq0: X_n = 3\} \]
      \item The probability that at 1 past the Stopping Time $T$, $X_n = 2$, given that the
        value of $X_n$ at the Stopping Time is 3, and the Stopping Time is finite
        \[ P(X_{T+1}=2 | X_T=3, T < \infty) \]
  }
   \dfn{ Stopping Time }{
     A random variable $T \in [0, \infty]$ is said to be a \underline{Stooping Time} if for any $n \geq 0$
     there exists  a set $C_n$ such that
     \[ \{T = n \} = \{ (X_0, \dots , X_n) \in C_n \} \]
     When $C_n$ denotes the set of sequences $(x_1, \dots x_n)$ where $x_1 \neq x, \dots, x_{n-1} \neq x$
     and $x_n = x$
     \ex{$\{T_x = n\}$} {
      \[\{T_x = n\} = \{ X_1 \neq x, X_2 \neq x, \dots, X_n = x\} \]
       So, $X_n = x$
     }
   \nt{
     To verify a \underline{Stopping Time},  justify that $\{T=n\}$ only involves the history
     of the MC up to time n.
   }
   }

   \dfn{ Hitting Time }{
     For any set $A \subset S$, we denote the \underline{hitting time} by
     \[ \tau_A = inf \{n \geq 0: X_n \in A \} \]
     Then $\tau$ is also a \underline{Stopping Time} because
     \[ \{ \tau_A = n \} = \{ X_0 \notin A, X_{n-1} \notin A, X_n \in A \} = \{ (X_0, \dots, X_n) \in C_n \} \]
   }
   \dfn{ Strong Markov Peroprty }{
     For a time-homogeneous MC and and set A,B and any $\mu$, and any stopping time T
     \[ P_\mu (  (X_{T+1}, \dots ) \in A | X_T=x, T > \infty , (X_0, \dots, X_T) \in B = P_x( (X_1, X_2 \dots ) \in A) \]
     The probability that the future behaves like A, is the same as a MC starting fresh
   }
  \dfn{ Transience }{
    If $\rho_{xx} < 1$
    \[ P_x(T_x^k = \infty \text{ for some k } = 1 \]
    \[ P_x(T_x^k < \infty \text{ for all k } = 0 \]
    $N_x < \infty$
    To prove that a state is transient, show that it has a P > 0 of escaping frfom the state, and there
    is a recurrent state with p=1
  }
  \dfn{ Recurrence }{
    If $\rho_{xx} = 1$
    \[ P_x(T^k_x < \infty, \text{ for all  } k \geq 1) = 1 \]
    \[ P_x(T^k_x = \infty, \text{ for all  } k \geq 1) = 0 \]
      \[N_x = \infty \]
      Recurrence is contagious, i.e. any state that communicates with a recurrent state is recurrent
  }
    \dfn{ Irreducible }{
      $A \subset S$ is irredicible if $x \leftrightarrow y$ for all $x,y \in A$
    }
    \dfn{ Close }{
      We say that $B \subset S$ is closed if for any $x \in B, y \notin B$, we have $x \not\rightarrow y$ 

      $R_x$ is called the "Communicating Class of x"
    }
\end{document}
