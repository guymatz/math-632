\documentclass{report}

\input{.tex/preamble}
\input{.tex/macros}
\input{.tex/letterfonts}

\title{
  \Huge{Math 632 - Stochastic Processes}
  \\
  Notes
}
\author{\huge{Guy Matz}}
\date{}
\begin{document}

\chapter{Markov Chains}

\section{Definitions \& Examples}
  \begin{itemize}
    \item A Discrete Time Markov chain with transition matrix $p(i, j)$
\[ P\left(X_{n+1}=j \mid X_n=i, X_{n-1}=i_{n-1}, \ldots, X_0=i_0\right)=p(i, j) \]

    Where $X_{n+1} = j$ means being at "position" $j$ at "time" $n+1$ 
    \item Since Markov, we have
      \[ p(i, j) = P(X_{n+1} = j  | X_n = i) \]

  \end{itemize}
  

\section{Multistep Transition Probabilities}

  \begin{itemize}
    \item $p^m(i,j) = P(X_{n+m} = j | X_n = i)$
  \end{itemize}

\section{Classification of States}
  \begin{itemize}
    \item $T_y$: Time of the first return to $y$ ($n$ is the number of moves?)
      \[ T_y = \text{min}\{n \geq 1 : X_n = y\} \]
    \item The Probability $X_n$ returns to $y$ when it starts at $y$
      \[ \rho_{yy} = P_y(T_y < \infty) \]
      The probability the state will return to $y$ within an inifite period of time
      \item \textbf{$T$ is a stopping time} if the iccurence (or nonoccurence) 
        of the event "we stop at time $n$", $\{T = n\}$, can be determined
        by looking at the values of the proess up to that time: $X_0, \dots, X_n$

      \item \textbf{Strong Markov Peroperty}: Suppose T is a stopping time.
        Given that $T=n$ and $X_T = y$, and any other inforamation about
        $X_0,\dots,X_T$ is irrelevent for predicting the future, and $X_{T+k},
        k \geq 0$ behaves like the Markov chain with initial state $y$

      \item \textbf{Transient}: A state that, after some point, is never visited
      \item \textbf{Recurrent}: A state that, after some point, continually recurs
      \item \textbf{$x$ communicates with $y$ ($x \rightarrow y$)} if there is a
        positive probability of reaching $y$ starting from $x$
      \item  A set $A$ is  \textbf{Closed} it it is impossible to get out, 
        i.e., if $i \in A$ and $j \notin A$, then $p(i,j) = 0$.
      \item A set $B$ is \textbf{irreducible} if whenever $i, j \in B, i$
        communicates with $j$.
      \item If $C$ is a finite closed and irreducible set, then all states
        in $C$ are recurrent


  \end{itemize}

\chapter{Renewal Processes}
\section{Laws of Large Numbers}
\dfn{Inter-arrival Times}{$X_n$: Random Variable for one occurence}
\dfn{Epoch Times}{$S_n$: Time for $n$ arrivals}
\dfn{$N(t)$}{Number of arrivals in time $t$}

\end{document}
